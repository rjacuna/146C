\documentclass{article}
\usepackage{fontspec}

% Used to embed Sage code in latex
\usepackage{sagetex}

% Math Environment
\usepackage{euler}        % Euler font
\usepackage{amsmath}      % Math macros
\usepackage{amssymb}      % Math symbols
\usepackage{unicode-math} % Unicode support

% Physics Environment
\usepackage{physics}


\usepackage[makeroom]{cancel} % Used to cancel terms in algebraic equations
\usepackage{ulem} % Different underline environments
\usepackage{polynom} %Polynomial long division

% Typesetting Rules
\setlength\parindent{0em}
\setlength\parskip{0.618em}
\usepackage[a4paper,lmargin=1in,rmargin=1in,tmargin=1in,bmargin=1in]{geometry}
\setmainfont[Mapping=tex-text]{Helvetica Neue LT Std 45 Light}

% Common Macros
\newcommand\N{\mathbb{N}}
\newcommand\Z{\mathbb{Z}}
\newcommand\R{\mathbb{R}}
\newcommand\C{\mathbb{C}}
\newcommand\A{\mathbb{A}}
\def\res{\mathop{\text{Res}}\limits}

% Color
\usepackage[dvipsnames]{xcolor}
\usepackage{pagecolor}
\definecolor{DeepCyan}{HTML}{006969}
\definecolor{DeepRed}{HTML}{690000}
\pagecolor{DeepCyan}
\color{white}



\begin{document}

\begin{center}
  146C --- 4

  RJ Acuña

  (862079740)
\end{center}\vspace{1.618em}
\subsection*{1} Factor the following polynomials:
\paragraph{(1)} $t^2-x^2 = (t-x)(t+x)$
\paragraph{(2)} $t^2 +tx -2x^2 = (t-x)(t+2x)$
\paragraph{(3)} $t^2 +6tx +9x^2 = (t+3x)^2$
\subsection*{2} Write the following second order differential operators as products of first order
operators:
\paragraph{(1)} $\pdv[2]{}{t}-\pdv[2]{}{x} = (\pdv{t}-\pdv{x})(\pdv{t}+\pdv{x})$
\paragraph{(2)} $\pdv[2]{t} +\pdv{}{x}{t} -2\pdv[2]{}{x} = (\pdv{t}-\pdv{x})(\pdv{t}+2\pdv{x})$
\paragraph{(3)} $\pdv[2]{}{t} +6\pdv{}{x}{t} +9\pdv[2]{}{x} = (\pdv{t}+3\pdv{x})^2$
\subsection*{3} Solve for the functions $a$, $b$, and $c$
\paragraph{(1)} $(\pdv{t}-\pdv{x})a = 0$ \uwave{slu.} $a(t,x) =
\alpha(t+x)\quad \lozenge$
\paragraph{(2)} $(\pdv{t}+2\pdv{x})b = 0$ \uwave{slu.} $b(t,x) =
\beta(2t-x)\quad \lozenge$
\paragraph{(3)} $(\pdv{t}+3\pdv{x})c = 0 $\uwave{slu.} $c(t,x) =
\gamma(3t-x)\quad \lozenge$
\newpage
\subsection*{4} Solve for the functions $u$, $v$, and $w$
\paragraph{(1)} $(\pdv{t}+\pdv{x})u = a$

\uwave{slu.}

Let $\tau = t-x$, and $y = t+x$,

Then $u_t = u_{\tau} +u_y$ and $u_x = -u_{\tau}+ u_y$, so
\[u_t+u_x = 2u_y = \alpha(y) \implies u = f(y) + g(\tau)\]
where $f'(y) = \frac{1}{2}\alpha(y)$.

Therefore, \[u(t,x) = f(t+x)+g(t-x)\quad \blacklozenge\]
\paragraph{(2)} $(\pdv{t}-\pdv{x})v = b$

\uwave{slu.}

Let $\tau = t+x$, and $y = 2t-x$,

Then $v_t = v_{\tau} +2v_y$ and $v_x = v_{\tau}- v_y$, so
\[v_t-v_x = v_y = \beta(y) \implies v = f(y) + g(\tau)\]
where $f'(y) = \beta(y)$.

Therefore, \[v(t,x) = f(t+x)+g(2t-x)\quad \lozenge\]

\paragraph{(3)} $(\pdv{t}+3\pdv{x})w = c$

Let $y = 3t-x$,

Then $w_t = \tau_tw_{\tau} +3w_y$ and $w_x = \tau_xw_{\tau}- w_y$, so
\[w_t+3w_x = (\tau_t+3\tau_x)w_{\tau} = \gamma(y)\]

Let $\tau = t$, then $w_{\tau} = \gamma(y) \implies w =
\tau\gamma(y)+g(y)$

Plug-in $\tau$ and $y$ and verify,
\[w_t = \gamma(3t-x) +3t\gamma'(3t-x) +3g'(3t-x)\]
\[w_x = -t\gamma'(3t-x) -g'(3t-x) \]
\[\implies w_t+3w_x = \gamma(3t-x)\]

So, \[w(t,x) = t\gamma(3t-x)+g(3t-x)\quad\blacklozenge\]
\newpage
\subsection*{5} Use Problem 4 to solve the following second order differential equations
\paragraph{(1)} $\pdv[2]{u}{t}-\pdv[2]{u}{x} = 0$

\uwave{slu.}

Factor the operator $(\pdv{t}-\pdv{x})(\pdv{t}+\pdv{x})u = 0$.

Let $a = (\pdv{t}+\pdv{x})u$, to reduce (1) to $\begin{cases}
  (\pdv{t}-\pdv{x})a = 0\\ (\pdv{t}+\pdv{x})u = a\end{cases}$

By 4.(1), the solution is,
\[u(t,x) = f(t+x)+g(t-x)\quad \lozenge\]

\paragraph{(2)} $\pdv[2]{v}{t} +\pdv{v}{x}{t} -2\pdv[2]{v}{x} = 0$

\uwave{slu.}

Factor the operator $(\pdv{t}-\pdv{x})(\pdv{t}+2\pdv{x})v =0$

Let $b = (\pdv{t}-\pdv{x})v$, to reduce (1) to $\begin{cases}
  (\pdv{t}+2\pdv{x})b = 0\\ (\pdv{t}-\pdv{x})v = b\end{cases}$

By 4.(2), the solution is,
\[u(t,x) = f(t+x)+g(2t-x)\quad \lozenge\]

\paragraph{(3)} $\pdv[2]{w}{t} +6\pdv{w}{x}{t} +9\pdv[2]{w}{x} = 0$

\uwave{slu.}

Factor the operator
\[(\pdv{t}+3\pdv{x})^2w =0\]

Let $c = (\pdv{t}+3\pdv{x})w$, to reduce (1) to $\begin{cases}
  (\pdv{t}+3\pdv{x})c = 0\\ (\pdv{t}-\pdv{x})w = c\end{cases}$

By 4.(3), the solution is,
\[w(t,x) = t\gamma(3t-x)+g(3t-x)\quad \lozenge\]

\subsection*{6} 6. In each of the problems above, suppose that $f$ is any one of the three functions
$u, v,$ or $w$ above. Given that
\[f(0,x) =\phi(x)\text{ and } f_t(0,x) = \psi(x).\]
,solve for $f$ in terms of $\phi$ and $\psi$. You may wish to study how Strauss does this in his derivation
of the solution to the wave equation.

\uwave{slu.}
\paragraph{(1)} The equation 5.(1) is the wave equation with $c = 1$, thus
the solution with initial conditions is, \[u(x,t) =
  \frac{1}{2}[\phi(x+t)+\phi(x-t)] + \int_{x-t}^{x+t} \psi(s)
  \dd{s}\]
The solution is due to Jean-Baptiste le Rond d'Alembert in 1746.
\newpage
\paragraph{(2)} The equation in 5.(2) has solution
\[u(t,x) =  f(t+x)+g(2t-x)\]
Multiplying the argument of $g$ by $-1$ doesn't change the form of the
solution as $g$ is arbitrary, so,
\[u(t,x) =  f(x+t)+g(x-2t)\]
Differentiating with respect to $t$, we get
\[u_t(t,x) =  f'(x+t)-2g'(x-2t)\]
Plugging in $0$, we get
\[u(0,x) =  f(x)+g(x) = \phi(x) \text{ and } u_t(0,x) =
  f'(x)-2g'(x) = \psi(x)\]
Then
\[\phi'(x) = f'(x)+g'(x) \text{ and } \psi(x) = f'(x)-2g'(x)\]

Solving the system for $f'(x)$, and $g'(x)$ we get
\[f'(x) = \frac{1}{3}(2\phi'(x) + \psi(x)) \text{ and } g'(x) = \frac{1}{3}(\phi'(x) -\psi(x))\]
So,
\[f(x) = \frac{1}{3}\left(  2\phi(x) -\int_0^x\psi(s) \dd{s}\right) \text{ and } g(x) =\frac{1}{3}\left(
  \phi(x) -\int_0^x \psi(s) \dd{s}\right)\]
Plugging $f$ and $g$ into the solution,
\[v(t,x) = \frac{1}{3}\left(  2\phi(x+t) +\int_0^{x+t}\psi(s) \dd{s}\right) +\frac{1}{3}\left(
    \phi(x-2t) -\int_0^{x-2t} \psi(s) \dd{s}\right)\]
Rearranging,
\[v(t,x) = \frac{1}{3}(2\phi(x+t) +\phi(x-2t)) +\frac{1}{3} \int_{x-2t}^{x+t}\psi(s)
  \dd{s}\]
\paragraph{(3)} The equation in 4.(3), has solution in the form,
\[w(t,x) = tf(x-3t)+g(x-3t)\]
\[w_t(t,x) = f(x-3t)-3tf'(x-3t)-3g'(x-3t)\]
\[w(0,x) = g(x) = \phi(x) \text{ and } w_t(0,x) = f(x)-3g'(x) = \psi(x)\]
So,
\[\phi(x) = g(x) \text{ and } \psi(x) = f(x) -3\phi'(x)\]
Solving for $f$, and $g$,
\[f(x) = \psi(x) +3\phi'(x)\text{ and } g(x) = \phi(x)\]
\[w(t,x) = t\left(\psi(x-3t) +3\phi'(x-3t)\right) + \phi(x-3t)\]
\newpage
Since the solution is a bit different I want to verify,
\[w_t(t,x) = \psi(x-3t) +3\phi'(x-3t) -3t\psi'(x-3t) -9t\phi''(x-3t)  -3\phi'(x-3t)\]
\[w_t(t,x)= \psi(x-3t) -3t\psi'(x-3t) -9t\phi''(x-3t) \implies w_t(0,x) = \psi(x) \text{ and } w(0,x) =\phi(x)\]
\[w_{tt}(t,x) =-3\psi'(x-3t)-3\psi'(x-3t)-9\phi''(x-3t)+9t\psi''(x-3t) +27t\phi'''(x-3t)\]
\[w_{tt}(t,x) =
  \color{green}-6\psi'(x-3t)
  \color{Goldenrod}-9\phi''(x-3t)
  \color{Peach}+9t\psi''(x-3t) \color{cyan}+27t\phi'''(x-3t)\]
\[w_{tx}(t,x) = \psi'(x-3t) -3t\psi''(x-3t) -9t\phi'''(x-3t)\]
\[6w_{tx}(t,x) = \color{green}6\psi'(x-3t)
  \color{Peach}-18t\psi''(x-3t) \color{cyan}-54t\phi'''(x-3t)\]
\[w_x(t,x) = t\left(\psi'(x-3t) +3\phi''(x-3t)\right) + \phi'(x-3t)\]
\[w_{xx}(t,x) = t\left(\psi''(x-3t) +3\phi'''(x-3t)\right) +
  \phi''(x-3t) = t\psi''(x-3t) +3t\phi'''(x-3t) +\phi''(x-3t)\]
\[9w_{xx}(t,x) = \color{Peach}9t\psi''(x-3t) \color{cyan}+27t\phi'''(x-3t) \color{Goldenrod}+9\phi''(x-3t)\]
So,
\[w_{tt} +6w_{tx}+9w_{xx} = 0\]
So the solutions (1),(2), and (3) solve the equations in
consideration for the corresponding initial conditions$\quad \blacklozenge$

\end{document}
%%% Local Variables:
%%% mode: latex
%%% TeX-master: t
%%% End:
