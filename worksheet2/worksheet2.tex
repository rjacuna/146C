\documentclass{article}
\usepackage{fontspec}

% Used to embed Sage code in latex
\usepackage{sagetex}

% Math Environment
\usepackage{euler}        % Euler font
\usepackage{amsmath}      % Math macros
\usepackage{amssymb}      % Math symbols
\usepackage{unicode-math} % Unicode support

% Physics Environment
\usepackage{physics}


\usepackage[makeroom]{cancel} % Used to cancel terms in algebraic equations
\usepackage{ulem} % Different underline environments
\usepackage{polynom} %Polynomial long division

% Typesetting Rules
\setlength\parindent{0em}
\setlength\parskip{0.618em}
\usepackage[a4paper,lmargin=1in,rmargin=1in,tmargin=1in,bmargin=1in]{geometry}
\setmainfont[Mapping=tex-text]{Helvetica Neue LT Std 45 Light}

% Common Macros
\newcommand\N{\mathbb{N}}
\newcommand\Z{\mathbb{Z}}
\newcommand\R{\mathbb{R}}
\newcommand\C{\mathbb{C}}
\newcommand\A{\mathbb{A}}
\def\res{\mathop{\text{Res}}\limits}

% Color
\usepackage[dvipsnames]{xcolor}
\usepackage{pagecolor}
\definecolor{DeepCyan}{HTML}{006969}
\definecolor{DeepRed}{HTML}{690000}
\pagecolor{DeepCyan}
\color{white}



\begin{document}

\begin{center}
  146C --- 2

  RJ Acuña

  (862079740)
\end{center}\vspace{1.618em}

Let's solve the
problem $a,b \in \R$,
\[\begin{cases}u_{tt} +au_{tx}+bu_{xx} = 0\\ u(0,x)= \phi(x)\text{ and
    }u_t(0,x) = \psi(x)\end{cases}\]

$t^2+atx+bx^2 = 0 \implies t = \frac{-ax\pm\sqrt{(ax)^2 - 4bx^2}}{2} =
\frac{-a \pm \sqrt{a^2-4b}}{2}x \implies t^2+atx+bx^2 = \left(t +
  \frac{a + \sqrt{a^2-4b}}{2}x \right)\left(t + \frac{a -
    \sqrt{a^2-4b}}{2}x  \right)$
From the factorization we get two cases,

In the case that
$a^2-4b = 0$, we can factor the operator and see that the solution for
$u$ in the following system, corresponds to the solution to the
original equation,
\[\begin{cases} \left(\pdv{t} +
      \frac{a}{2}\pdv{x} \right)d = 0\\ \left(\pdv{t} +
      \frac{a}{2}\pdv{x}  \right)u = d\end{cases}\]

Then, $d(x,t) = f(x-\frac{a}{2}t)$, solves the first equation in the system.

Let $y = x-\frac{a}{2}t$, then $u_t = \tau_tu_{\tau}-\frac{a}{2}u_y$, and
$u_x = \tau_xu_{\tau} +u_y$.So,
\[u_t+\frac{a}{2}u_x =(\tau_t+\frac{a}{2}\tau_x)u_{\tau} = f(y) \]

Let $\tau = t$, so $\tau_t+\frac{a}{2}\tau_x = 1$,
\[u_{\tau} = f(y) \implies u(\tau,y) = \tauf(y) +g(y) \]

So,
\[u(t,x) = tf(x-\frac{a}{2}t) +g(x-\frac{a}{2}t)\]
\[u_(t,x) = f(x-\frac{a}{2}t) -t\frac{a}{2}f'(x-\frac{a}{2}t) -\frac{a}{2}g'(x-\frac{a}{2}t)\]
Now,
\[u(0,x) = g(x) = \phi(x)\text{ and } u_t(0,x) = f(x)
  -\frac{a}{2}g'(x) = \psi(x)\]
So,
\[f(x) = \psi(x)+\frac{a}{2}\phi'(x)\text{ and }g(x) = \phi(x)\]
Therefore,
\[u(t,x) = t\left( \psi(x-\frac{a}{2}t)
    +\frac{a}{2}\phi'(x-\frac{a}{2}t)\right)+\phi(x-\frac{a}{2}t).\]

In that case you can check that if $a = 0 $, then $b = 0$. So,
\[u(t,x) = t\left( \psi(x) \right)+\phi(x),\]
which is the same result that we would've gotten by integrating
$u_{tt} = 0.$

In the case that $a^2-4b \neq 0$, we can factor the operator, and see
that the solution for $u$ in the following system, correspond to
the solution to the original equation.
\[\begin{cases} \left(\pdv{t} +
      \frac{a + \sqrt{a^2-4b}}{2}\pdv{x} \right)d = 0\\ \left(\pdv{t} + \frac{a -
        \sqrt{a^2-4b}}{2}\pdv{x}  \right)u = d\end{cases}\]
So, \[u(t,x) = f\left(x - \frac{a+\sqrt{a^2-4b}}{2}t \right) + g\left(x -\frac{a -
      \sqrt{a^2-4b}}{2}t\right)\]
\[u(0,x) = f(x)+g(x) = \phi(x)\implies \phi'(x) = f'(x)+g'(x)\]
\[u_t(t,x) = - \frac{a+\sqrt{a^2-4b}}{2}f'\left(x - \frac{a+\sqrt{a^2-4b}}{2}t \right) - \frac{a-\sqrt{a^2-4b}}{2}g'\left(x -\frac{a -
      \sqrt{a^2-4b}}{2}t\right)\]
\[u_t(0,x) = - \frac{a+\sqrt{a^2-4b}}{2}f'\left(x\right) -
  \frac{a-\sqrt{a^2-4b}}{2}g'\left(x\right) = \psi(x) \implies
  \psi(x) = - \frac{a+\sqrt{a^2-4b}}{2}f'\left(x\right) -
  \frac{a-\sqrt{a^2-4b}}{2}g'\left(x\right)\]

We have the following linear system of equations
\[\begin{cases}
  \hspace{4.8em}f'(x)\,\,+&g'(x) = \phi'(x)\\
  - \frac{a+\sqrt{a^2-4b}}{2}f'\left(x\right) -
  \frac{a-\sqrt{a^2-4b}}{2}&g'\left(x\right) = \psi(x)\end{cases}
\implies \begin{pmatrix}1&1 \\ - \frac{a+\sqrt{a^2-4b}}{2} & -
  \frac{a-\sqrt{a^2-4b}}{2}\end{pmatrix}\begin{pmatrix}
  f'(x)\\g'(x)\end{pmatrix}
= \begin{pmatrix}\phi'(x)\\\psi(x)\end{pmatrix}\]
\[\begin{vmatrix}1&1 \\ - \frac{a+\sqrt{a^2-4b}}{2} & -
  \frac{a-\sqrt{a^2-4b}}{2}\end{vmatrix} =-
  \frac{a-\sqrt{a^2-4b}}{2} +
  \frac{a+\sqrt{a^2-4b}}{2} = \sqrt{a^2-4b} \]
\[\begin{pmatrix}1&1 \\ - \frac{a+\sqrt{a^2-4b}}{2} & -
  \frac{a-\sqrt{a^2-4b}}{2}\end{pmatrix}^{-1} = \frac{1}{\sqrt{a^2-4b}}\begin{pmatrix}-
  \frac{a-\sqrt{a^2-4b}}{2}&-1 \\  \frac{a+\sqrt{a^2-4b}}{2} &
  1\end{pmatrix} = \begin{pmatrix}-
  \frac{a}{2\sqrt{a^2-4b}} + \frac{1}{2}&-\frac{1}{\sqrt{a^2-4b}} \\
  \frac{a}{2\sqrt{a^2-4b}}+\frac{1}{2} &
  \frac{1}{\sqrt{a^2-4b}}\end{pmatrix}\]
\[\implies \begin{pmatrix}
  f'(x)\\g'(x)\end{pmatrix}
= \begin{pmatrix}-
  \frac{a}{2\sqrt{a^2-4b}} + \frac{1}{2}&-\frac{1}{\sqrt{a^2-4b}} \\
  \frac{a}{2\sqrt{a^2-4b}}+\frac{1}{2} &
  \frac{1}{\sqrt{a^2-4b}}\end{pmatrix}\begin{pmatrix}\phi'(x)\\\psi(x)\end{pmatrix} \]
So,
\[f'(x) = -\left(
  \frac{a}{2\sqrt{a^2-4b}} - \frac{1}{2}\right)\phi'(x)-\frac{1}{\sqrt{a^2-4b}}\psi(x)\]
\[f(x) = -\left(
  \frac{a}{2\sqrt{a^2-4b}} - \frac{1}{2}\right)\phi(x)-\frac{1}{\sqrt{a^2-4b}}\int_0^x\psi(s)\dd{s}\]
\[g'(x) = \left(
  \frac{a}{2\sqrt{a^2-4b}} + \frac{1}{2}\right)\phi'(x)+\frac{1}{\sqrt{a^2-4b}}\psi(x)\]
\[g(x) = \left(
  \frac{a}{2\sqrt{a^2-4b}} + \frac{1}{2}\right)\phi(x) +\frac{1}{\sqrt{a^2-4b}}\int_0^x\psi(s)\dd{s}\]

Then,
\begin{align*}
  u(t,x)
  &= -\left(\frac{a}{2\sqrt{a^2-4b}}
    -\frac{1}{2}\right)\phi\left(x - \frac{a+\sqrt{a^2-4b}}{2}t \right)
    -\frac{1}{\sqrt{a^2-4b}}\int_0^{x - \frac{a+\sqrt{a^2-4b}}{2}t }\psi(s)\dd{s}\\
  &+\left(\frac{a}{2\sqrt{a^2-4b}}
    + \frac{1}{2}\right)\phi\left(x -\frac{a-\sqrt{a^2-4b}}{2}t \right)
    +\frac{1}{\sqrt{a^2-4b}}\int_0^{x - \frac{a-\sqrt{a^2-4b}}{2}t
    }\psi(s)\dd{s}
\end{align*}
Therefore,
\begin{align*}
  u(t,x)
  &= \frac{1}{2}\left( \phi\left(x - \frac{a+\sqrt{a^2-4b}}{2}t
    \right) + \phi\left(x -\frac{a-\sqrt{a^2-4b}}{2}t \right)\right)\\
  &+\frac{a}{2\sqrt{a^2-4b}}\left(\phi\left(x
    -\frac{a-\sqrt{a^2-4b}}{2}t \right) - \phi\left(x
    -\frac{a+\sqrt{a^2-4b}}{2}t \right)  \right)\\
  &+\frac{1}{\sqrt{a^2-4b}}\int_{x - \frac{a+\sqrt{a^2-4b}}{2}t }^{x - \frac{a-\sqrt{a^2-4b}}{2}t }\psi(s)\dd{s}
\end{align*}
In the case $a= 0 \text{ and }b < 0$,  putting
$b= c^2$, you can check it's exactly the same solution found in Strauss p.36 equation (8).
\newpage
If $a^2-4b < 0$ we don't get a real solution. However, any linear
combination solves the problem. Since the operator is linear, it's
real and imaginary parts must solve the problem and any combinations
of them as well,
\[\Re [u(t,x)] = \frac{u(t,x)+\overline{u(t,x)}}{2} \text{ and } \Im[
  u(t,x)] =  \frac{u(t,x)-\overline{u(t,x)}}{2i}\]

Let $u(t,x) = p(t,x) +iq(t,x)$, then $u_t(t,x) = p_t(t,x)+q_t(t,x)$

If there exists a linear combination of the real
and imaginary parts of $u$ that solves the IVP, that's our solution.

let $\sqrt{a^2-4b} = di$,
\begin{align*}
    u(t,x)
  &= \frac{1}{2}\left( \phi\left(x - \frac{a+di}{2}t
    \right) + \phi\left(x -\frac{a-di}{2}t \right)\right)\\
  &+\frac{a}{2di}\left(\phi\left(x
    -\frac{a-di}{2}t \right) - \phi\left(x
    -\frac{a+di}{2}t \right)  \right)\\
  &+\frac{1}{di}\int_{x - \frac{a+di}{2}t }^{x - \frac{a-di}{2}t }\psi(s)\dd{s}
\end{align*}
Verify $u(0,x) = \phi(x)$,
\begin{align*}
    u(0,x)
  &= \frac{1}{2}\left( \phi\left(x
    \right) + \phi\left(x \right)\right)
  +\frac{a}{2di}\left(\phi\left(x
    \right) - \phi\left(x
    \right)  \right)
  +\frac{1}{di}\int_{x }^{x }\psi(s)\dd{s} = \phi(x)
\end{align*}

Compute $u_t$,
\begin{align*}
  u_t(t,x)
  &= -\left(\frac{a+di}{4}\right) \phi'\left(x - \frac{a+di}{2}t
    \right)- \left(\frac{a-di}{4}\right)  \phi'\left(x -\frac{a-di}{2}t \right)\\
  &\color{green}+\left(  \frac{a}{2di}\frac{a+di}{2}\right)\color{white} \phi'\left(x
    -\frac{a+di}{2}t \right)
    \color{green}-\left(\frac{a}{2di}\frac{a-di}{2}\right)\color{white}\phi'\left(x
    -\frac{a-di}{2}t \right)
    \\
  &\color{cyan}-\left(  \frac{1}{di}\frac{a-di}{2}\right)\color{white}\psi\left(x -
    \frac{a-di}{2}t \right)\color{cyan}
  +\left(  \frac{1}{di}\frac{a+di}{2}\right)\color{white}\psi\left(x -
    \frac{a+di}{2}t \right)\\
  &= -\left(\frac{a}{4}+\frac{d}{4}i\right) \phi'\left(x - \frac{a+di}{2}t
    \right)- \left(\frac{a}{4}-\frac{d}{4}i\right)  \phi'\left(x -\frac{a-di}{2}t \right)\\
  &\color{green}+\left(\frac{a}{4}-\frac{a^2}{4d}i\right) \color{white}\phi'\left(x
    -\frac{a+di}{2}t \right)  \color{green}
    +\left(\frac{a}{4} +\frac{a^2}{4d}i\right)\color{white}\phi'\left(x
    -\frac{a-di}{2}t \right)
    \\
  &\color{cyan}+\left(  \frac{1}{2}+\frac{a}{2d}i\right)\color{white}\psi\left(x -
    \frac{a-di}{2}t \right) \color{cyan}
  +\left(  \frac{1}{2}-\frac{a}{2d}i\right)\color{white}\psi\left(x -
    \frac{a+di}{2}t \right)
\end{align*}

Verify $u_t(0,x) = \psi(x)$
\begin{align*}
  u_t(0,x)&= -\left(\frac{a}{4}+\frac{d}{4}i\right) \phi'\left(x
            \right)- \left(\frac{a}{4}-\frac{d}{4}i\right)  \phi'\left(x  \right)\\
  &\color{green}+\left(\frac{a}{4}-\frac{a^2}{4d}i\right) \color{white}\phi'\left(x
     \right)  \color{green}
    +\left(\frac{a}{4} +\frac{a^2}{4d}i\right)\color{white}\phi'\left(x
     \right)
    \\
  &\color{cyan}+\left(  \frac{1}{2}+\frac{a}{2d}i\right)\color{white}\psi\left(x  \right) \color{cyan}
  +\left(  \frac{1}{2}-\frac{a}{2d}i\right)\color{white}\psi\left(x
    \right) = \psi(x)
\end{align*}

Verify $w(t,x) = \Re[u(t,x)] + \Im[u(t,x)]$ solves the IVP,

\[w(0,x) = p(0,x)+q(0,x) = \phi(x) \text{ and } w_t(t,x) = p_t(x,t) + q_t(x,t)\]

We've shown,
\[u_t(0,x) = \psi(x) \in \R \implies q_t(0,x) = 0 \text{ and } w_t(0,x) = p_t(0,x) = \psi(x)\]

So, for $a^2-4b<0$ the solution is, \[w(t,x) = \Re[u(t,x)] +
  \Im[u(t,x)]\]
\end{document}
%%% Local Variables:
%%% mode: latex
%%% TeX-master: t
%%% End:
