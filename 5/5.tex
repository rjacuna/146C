\documentclass{article}
\usepackage{fontspec}

% Used to embed Sage code in latex
\usepackage{sagetex}

% Math Environment
\usepackage{euler}        % Euler font
\usepackage{amsmath}      % Math macros
\usepackage{amssymb}      % Math symbols
\usepackage{unicode-math} % Unicode support

% Physics Environment
\usepackage{physics}


\usepackage[makeroom]{cancel} % Used to cancel terms in algebraic equations
\usepackage{ulem} % Different underline environments
\usepackage{polynom} %Polynomial long division

% Typesetting Rules
\setlength\parindent{0em}
\setlength\parskip{0.618em}
\usepackage[a4paper,lmargin=1in,rmargin=1in,tmargin=1in,bmargin=1in]{geometry}
\setmainfont[Mapping=tex-text]{Helvetica Neue LT Std 45 Light}

% Common Macros
\newcommand\N{\mathbb{N}}
\newcommand\Z{\mathbb{Z}}
\newcommand\R{\mathbb{R}}
\newcommand\C{\mathbb{C}}
\newcommand\A{\mathbb{A}}
\def\res{\mathop{\text{Res}}\limits}

% Color
\usepackage[dvipsnames]{xcolor}
\usepackage{pagecolor}
\definecolor{DeepCyan}{HTML}{006969}
\definecolor{DeepRed}{HTML}{690000}
\pagecolor{DeepCyan}
\color{white}


\begin{document}

\begin{center}
  146C --- 5

  RJ Acuña

  (862079740)
\end{center}\vspace{1.618em}

Diffusion Equation (DE),
\[u_t = ku_{xx}\, .\]

Wave Equation (WE),
\[u_{tt} = k u_{xx}\, .\]

\paragraph{1} Suppose that $u$ is the solution to (DE) on the closed interval $[0, l]$. Suppose that
$\phi$ is a continuous function on $[0, l]$ and that
\[u(0,x) = \phi(x)\text{ and } u(t,0)= 0 = u(t,l)\]
Use the weak maximal property to argue that if $u$ and $v$ are two solutions to the above equation
with these initial conditions and boundary conditions, then for all
$(t, x)$ in $[0, T] × [0, l]$,
\[u(t,x) = v(t,x).\]

\uwave{pf.}

$u, $ and $v$ solve (DE). Then,
\[u_t = ku_{xx}\text{ and }v_t = kv_{xx} \implies (u-v)_t =
  k(u-v)_{xx}.\]
\[u(0,x) = \phi(x) = v(0,x) \implies (u-v)(0,x) = 0.\]
\[u(t,0) = v(t,0) = 0 = v(t,l) = u(t,l) \implies (u-v)(t,0) = 0 = (u-v)(t,l).\]

Therefore, $(u-v)(t,x) \equiv 0$, for $(t,x)$ on the boundary of the
semi-infinite strip $[0,\infty)\times[0,l]$. By the weak maximum
principle,the function $u-v$ has
it's maximum on the boundary. Since, solutions of (DE) are
concentrations, or temperatures, there are no negative solutions if
one uses an absolute scale, i.e. Kelvin. Therefore, everywhere on that
domain $(u-v)(t,x) \equiv 0$, so

\[u(t,x) = v(t,x)\quad \blacksquare\]

\paragraph{2} Suppose that $u$ is the solution to (DE) on $\R^2$. Suppose that $\phi$ is a positive
continuous function on $\R$ that is integrable on all of $\R$ (meaning the integral over $(−∞,∞)$ is
finite) and that
\[u(0, x) = \phi(x)\text{ and } \lim_{x\rightarrow -\infty} u(t, x) =
  0 = \lim_{x\rightarrow \infty} u(t, x)\]
Using the previous problem, argue that if $u$ and $v$ are two solutions to the above equation with
this initial condition, then for all $(t, x)$ in $[0, ∞) × (−∞,∞)$,
\[u(t, x) = v(t, x).\]

\uwave{pf.}

The boundary of the region $[0, ∞) × (−∞,∞)$, is $\{0\}× (−∞,∞)$, there
\[u(0,x) = \phi(x) = v(0, x)\implies (u-v)(0,x) = 0.\]
By the weak maximum property $(u-v)(t,x)\equiv 0$ as $u-v$ is
non-negative, for the same reasons as above. So,\[ u(t, x) = v(t,x)\quad \blacksquare\]

\paragraph{3} Solve Strauss 2.2 \# 1. Use this exercise to prove the following. Suppose that
$u$ is the solution to (WE) on $\R^2$. Suppose that $\phi$ is continuously differentiable and that $\psi$ is
continuous on $\R$ and that
\[u(0, x) = \phi(x), (2) u_t(0, x) = \psi(x).\]
Prove that if $u$ and $v$ are two solutions to the above equation with this initial condition, then
for all $(t, x)$ in $(−∞, ∞) × (−∞,∞)$,
$u(t, x) = v(t, x).$

\uwave{slu of Strauss 2.2 \# 1} See the first problem in my HW \#
3$\quad\blacksquare$

\uwave{pf. }
\[u_{tt} = k u_{xx}\text{ and }v_{tt} = kv_{xx} \implies (u-v)_{tt} = k(u-v)_{xx}\]
\[u(0, x) = \phi(x), u_t(0, x) = \psi(x), v(0, x) = \phi(x), \text{ and } v_t(0, x) = \psi(x)\]
\[\implies (u-v)(0,x)\equiv 0\text{ and } (u-v)_t(0,x) \equiv 0\]
By Strauss 2.2 \# 1, \[(u-v)(t,x) \equiv 0\implies u(x,t) =
  v(x,t)\quad\blacksquare\]

\paragraph{4} Suppose that $u$ is the solution to (DE) on the closed interval $[0, l]$ and suppose
that $v$ is the solution to (WE) on the closed interval $[0, l]$. Suppose further that
$v(t, 0) = v(t, l) = 0$ and $u(t, 0) = u(t, l) = 0$.
Given functions $T$ and $X$, find all solutions to these boundary value problems, where
\[u(t, x) = T(t)X(x)\text{ and }v(t, x) = T(t)X(x).\]
Note that you are only being asked to find a single solution for each
problem.

\uwave{slu.}

\[u_t = T'X = kTX'' = ku_{xx}\text{ and }v_{tt} = T''X = kTX''=v_{xx}\]
\[\implies \frac{T'}{T} = k\frac{X''}{X} = -\lambda \text{ and } \frac{T''}{T} =
  k\frac{X''}{X} = -\lambda\text{ both are constant}\]
\[\implies X''+ \frac{\lambda}{k} X = 0 \text{ and } \left(T'
  +\lambda T = 0\text{ or }  T''+ \lambda T = 0 \text{ respectively.}
\right)\]

The first ODE in $X$ is common to both equations, we know from a
previous course that with the boundary conditions as stated, the only
possible eigenvalues that yield non-trivial solutions are positive,
since $\lambda/k >0$, we have
\[X(x) = A\cos\left(\sqrt{\frac{\lambda}{k}} x\right) +
  B\sin\left(\sqrt{\frac{\lambda}{k}} x\right)\]
\[X(0) = 0 = A\cos(0) + B\sin(0) = A\]
\[X(l) = 0 = B\sin\left(\sqrt{\frac{\lambda}{k}}l\right) \]
If $B= 0$, we have the trivial solution, but we don't want trivial
solutions so,
\[\sin\left(\frac{\sqrt{\lambda} l}{\sqrt{k}}\right) = 0 \implies
  \frac{\sqrt{\lambda} l}{\sqrt{k}} = n\pi \implies
  \lambda =k \left(\frac{n\pi}{l}\right)^2\]
\[\implies X_n(x) = \sin\left( \frac{n\pi x}{l} \right)\text{ as the
  }\sqrt{k} \text{ cancels.}\]
\[\implies T'+k \left(\frac{n\pi}{l}\right)^2 T = 0 \implies T_n(t) = C_n\exp\left(-k\left(\frac{n\pi}{l}\right)^2 t\right)\]\[\implies T''+k
  \left(\frac{n\pi}{l}\right)^2 T = 0\implies T_n(t) =
  A_n\cos\left(\frac{\sqrt{k} n \pi
      t}{l}\right) + B_n\sin\left(\frac{\sqrt{k} n\pi t}{l}\right).\]
\[\implies u(t,x) = \sum_{n=1}^{\infty}C_n\exp\left(-k\left(
      \frac{n\pi}{l}\right)^2t \right)\sin\left( \frac{n\pi x}{l}
  \right),\]\[\text{ and }v(t,x) =  \sum_{n=1}^\infty\left(
    A_n\cos\left(\sqrt{k}\left(  \frac{ n\pi
        }{l}\right)t\right) + B_n\sin\left(\sqrt{k}\left(  \frac{n\pi}{l}\right) t\right) \right)\sin\left( \frac{n\pi x}{l}
  \right)\quad \blacklozenge\]

\paragraph{5} Suppose that $u$ is the solution to (DE) on the closed interval $[0, l]$ and
\[u(t, 0) = u(t, l) = 0.\]
Find all solutions to this boundary value problem, where

(1) $u(0, x) = 2 \sin\left( \frac{\pi x}{l} \right),$

(2) $u(0, x) = 5 \sin\left( \frac{3\pi x}{l} \right) +7 \sin\left( \frac{4\pi x}{l} \right).$

\uwave{slu. }

By 4,
\[u(0,x) = \sum_{n=1}^{\infty}C_n\exp\left(0\right)\sin\left( \frac{n\pi x}{l}
  \right) = \sum_{n=1}^{\infty}C_n\sin\left( \frac{n\pi x}{l}
  \right) \]

for (1) $C_1 = 2 $ and $ C_n = 0\quad  \forall n: n\neq 2$, then
\[\implies u(t,x) = 2\exp\left(-k\left(
      \frac{\pi}{l}\right)^2t \right)\sin\left( \frac{\pi x}{l}
  \right)\]
for (2) $C_3 = 5,C_4 = 7 $ and $ C_n = 0\quad  \forall n: n\neq 2$ and
$n \neq 7$, then
\[\implies u(t,x) = 5\exp\left(-k\left(
      \frac{3\pi}{l}\right)^2t \right)\sin\left( \frac{3\pi x}{l}
  \right)+7\exp\left(-k\left(
      \frac{4\pi}{l}\right)^2t \right)\sin\left( \frac{4\pi x}{l}
  \right)\quad \lozenge\]


\paragraph{6} Suppose that $v$ is the solution to (WE) on the closed
interval $[0, l]$, with $l$ positive
and
\[v(t, 0) = u(t, l) = 0.\]
Find all solutions to this boundary value problem, where

(1) $v(0, x) = 2 \sin\left( \frac{\pi x}{l} \right),$

(2) $v(0, x) = 5 \sin\left( \frac{3\pi x}{l} \right) +7 \sin\left( \frac{4\pi x}{l} \right).$


(3) Do you get unique solutions to the above problems? If not, what additional information
will give you unique solutions? Supply such information and find the solutions to these
problems with the additional information.

\uwave{slu. }

To simplify the form we let $\sqrt{k} = c$, the wave speed,
\[v(t,x) = \sum_{n=1}^\infty\left(
    A_n\cos\left(c\left(  \frac{ n\pi
        }{l}\right)t\right) + B_n\sin\left(c\left(  \frac{n\pi}{l}\right) t\right) \right)\sin\left( \frac{n\pi x}{l}
  \right).\]
Then, \[\text{ and }v(0,x) = \sum_{n=1}^\infty A_n\sin\left( \frac{n\pi x}{l}
  \right).\]

for (1) $A_1 = 2 $ and $ A_n = 0\quad  \forall n: n\neq 2$, then
\[\implies u(t,x) = 2\cos\left(c\left(
      \frac{\pi}{l}\right)t \right)\sin\left( \frac{\pi x}{l}
  \right) + \sum_{n=1}^\infty B_n\sin\left(c\left(  \frac{n\pi}{l}\right) t\right)\sin\left( \frac{n\pi x}{l}
  \right).\]
for (2) $C_3 = 5,C_4 = 7 $ and $ C_n = 0\quad  \forall n: n\neq 2$ and
$n \neq 7$, then
\[\implies u(t,x) = 5\cos\left(c\left(
      \frac{3\pi}{l}\right)t \right)\sin\left( \frac{3\pi x}{l}
  \right)+7\cos\left(c\left(
      \frac{4\pi}{l}\right)t \right)\sin\left( \frac{4\pi x}{l}
  \right)+ \sum_{n=1}^\infty B_n\sin\left(c\left(  \frac{n\pi}{l}\right) t\right)\sin\left( \frac{n\pi x}{l}\right)\]
for (3), no it's not enough information, we need to characterize the
$B_n$s by specifying $v_t(0,x) = \psi(x)$. If we let $v_t(0,x) = 0$,
then all of the $B_n$s above are $0$, and we get unique
solutions$\quad \lozenge$

\paragraph{7} Suppose that $u$ is the solution to (DE) on the closed interval $[0, l]$ and
\[u_x(t, 0) = u_x(t, l) = 0.\]
Find all solutions to this boundary value problem, where

(1) $u(0, x) = 5 \cos\left( \frac{\pi x}{l} \right),$

(2) $u(0, x) = 4 + 2\cos\left( \frac{3\pi x}{l} \right) +7 \sin\left( \frac{4\pi x}{l} \right).$

\uwave{slu.}

Since we have the condition,
\[u_x(t,0) = u_x(t,l) = 0.\]
It follows that,
\[X'(0)T(t) = X'(L)T(t) = 0 \implies X'(0) = X'(l) =0.\]

Consider the case $\lambda = 0$, then
\[X'' = 0 \implies X(x) = d_1x +d_2.\]
\[X''(x) = d_1 \text{ and }X''(0) = 0 \implies d_1 = 0.\]

However, $d_2$ is not determined by the boundary conditions, so
$\lambda = 0$ is an eigenvalue, with eigenfunction $X_0(x) = C_0.$

We know that there are no negative eigenvalues from a previous class.

Furthermore, the solution for the positive eigenvalues is the same,
with the difference that the eigenfunctions are $\cos(\frac{n\pi x}{l})$.Therefore,
\[u(t,x) = C_0 + \sum_{n=1}^{\infty} C_n\exp\left(-k\left(
      \frac{n\pi}{l}\right)^2t \right)\cos\left( \frac{n\pi x}{l}
  \right)\]

for (1) $C_1 = 5$, and $\forall n: n\neq 1 C_n = 0 \implies$
$C_1 = \frac{5l}{\pi}$. So,
\[u(t,x) = 5\exp\left(-k\left(
      \frac{\pi}{l}\right)^2t \right)\cos\left(\frac{n\pi x}{l}
  \right).\]

for (2) $C_0 = 4$, $C_3 = 2,$ and $C_4 = 7$, $\forall n: n\neq 3$ and
$n\neq 4, C_n = 0$. So,
\[ u(t,x) = 4 + 2\exp\left(-k\left(
      \frac{3\pi}{l}\right)^2t \right)\cos\left( \frac{3\pi x}{l}
  \right) + 7\exp\left(-k\left(
      \frac{4\pi}{l}\right)^2t \right)\cos\left( \frac{3\pi x}{l}
  \right)\quad \lozenge\]

\newpage
\paragraph{8} Suppose that $u$ is the solution to (WE) on the closed interval $[0, l]$ and
\[v_x(t, 0) = v_x(t, l) = 0.\]
Find all solutions to this boundary value problem, where

(1) $v(0, x) = 5 \cos\left( \frac{\pi x}{l} \right),$

(2) $v(0, x) = 4 + 2\cos\left( \frac{3\pi x}{l} \right) +7 \sin\left( \frac{4\pi x}{l} \right).$

(3) Do you get unique solutions to the above problems? If not, what additional information
will give you unique solutions? Supply such information and find the solutions to these
problems with the additional information.

\uwave{slu. }
The same way as number 7,

\[v(t,x) = A_0 +\sum_{n=0}^\infty \left(
    A_n\cos\left(c\left(  \frac{ n\pi
        }{l}\right)t\right) + B_n\sin\left(c\left(  \frac{n\pi}{l}\right) t\right) \right)\cos\left( \frac{n\pi x}{l}
  \right)\]

for (1) $C_1 = 5$, and $\forall n: n\neq 1 C_n = 0 \implies$
$C_1 = \frac{5l}{\pi}$. So,
\[v(t,x) = 5\cos\left(c\left(
      \frac{\pi}{l}\right)t \right)\cos\left(\frac{n\pi x}{l}
  \right) + \sum_{n=0}^\infty  B_n\sin\left(c\left(  \frac{n\pi}{l}\right) t\right) \cos\left( \frac{n\pi x}{l}
  \right) .\]

for (2) $C_0 = 4$, $C_3 = 2,$ and $C_4 = 7$, $\forall n: n\neq 3$ and
$n\neq 4, C_n = 0$. So,
\[v(t,x) = 4 + 2\cos\left(c\left(
      \frac{3\pi}{l}\right)t \right)\cos\left( \frac{3\pi x}{l}
  \right) + 7\cos\left(c\left(
      \frac{4\pi}{l}\right)t \right)\cos\left( \frac{4 \pi x}{l}
  \right) + \sum_{n=0}^\infty  B_n\sin\left(c\left(  \frac{n\pi}{l}\right) t\right) \cos\left( \frac{n\pi x}{l}
  \right).\]

for (3), no it's not enough information, we need to characterize the
$B_n$s by specifying $v_t(0,x) = \psi(x)$. If we let $v_t(0,x) = 0$,
then all of the $B_n$s above are $0$, and we get unique
solutions$\quad \lozenge$


\end{document}
%%% Local Variables:
%%% mode: latex
%%% TeX-master: t
%%% End:
