\documentclass{article}
\usepackage{fontspec}

% Used to embed Sage code in latex
\usepackage{sagetex}

% Math Environment
\usepackage{euler}        % Euler font
\usepackage{amsmath}      % Math macros
\usepackage{amssymb}      % Math symbols
\usepackage{unicode-math} % Unicode support

% Physics Environment
\usepackage{physics}


\usepackage[makeroom]{cancel} % Used to cancel terms in algebraic equations
\usepackage{ulem} % Different underline environments
\usepackage{polynom} %Polynomial long division

% Typesetting Rules
\setlength\parindent{0em}
\setlength\parskip{0.618em}
\usepackage[a4paper,lmargin=1in,rmargin=1in,tmargin=1in,bmargin=1in]{geometry}
\setmainfont[Mapping=tex-text]{Helvetica Neue LT Std 45 Light}

% Common Macros
\newcommand\N{\mathbb{N}}
\newcommand\Z{\mathbb{Z}}
\newcommand\R{\mathbb{R}}
\newcommand\C{\mathbb{C}}
\newcommand\A{\mathbb{A}}
\def\res{\mathop{\text{Res}}\limits}

% Color
\usepackage[dvipsnames]{xcolor}
\usepackage{pagecolor}
\definecolor{DeepCyan}{HTML}{006969}
\definecolor{DeepRed}{HTML}{690000}
\pagecolor{DeepCyan}
\color{white}



\begin{document}

\begin{center}
  Diffusion Equation

  RJ Acuña
\end{center}\vspace{1.618em}

The following equation looks deceptively simple.
\begin{equation}
  \color{Goldenrod} u_t = ku_{xx}
\end{equation}

It is not! As we
shall see.

First, we do the standard trick of passing to operators,
\[\color{Goldenrod}u_t = ku_{xx}\iff u_t -ku_{xx} = 0\iff (\partial_t
  -k\partial_x^2)u=0\]
So we obtain the following equation,
\begin{equation}
  \color{Goldenrod} (\partial_t
  -k\partial_x^2)u = 0
\end{equation}

Now, make the reasonable assumption that we can factor the operator,
\[\color{Goldenrod}(\partial_t
  -k\partial_x^2)u=0 \implies (\color{green}\partial_t^{\frac{1}{2}}\color{Goldenrod}
  {\color{white} +} k\partial_x)(\color{green}\partial_t^{\frac{1}{2}}
  \color{Goldenrod}{\color{cyan} -} k\partial_x)u=0\]

We need to make sense of $\color{green}\partial_t^{\frac{1}{2}}$, for
that we turn to Fractional Equations and Models by Sandev and Tomovski
page 31 formula (2.7). With suitable modifications we can see that

\begin{align*}
  \partial_{\color{cyan}t}^{\color{green}\frac{1}{2}} w(t,x)
  &= \frac{1}{\Gamma(1  -{\color{green}\frac{1}{2}})}\int_0^t(t-{\color{cyan}\tau})^{1-{\color{green}\frac{1}{2}}-1}
    w_{\color{cyan}\tau}({\color{cyan} \tau},x)\dd{\color{cyan}\tau}\\
  &=
    \frac{1}{\Gamma({\color{Goldenrod}\frac{1}{2}})}\int_0^t(t-{\color{cyan}
    \tau})^{-{\color{Goldenrod}\frac{1}{2}}}
    w_{\color{cyan}\tau}({\color{cyan} \tau},x)\dd{\color{cyan}\tau}\\
  &= \frac{1}{\color{Goldenrod} \sqrt{\pi}}\int_0^t
    \frac{w_{\color{cyan}\tau}({\color{cyan} \tau},x)}{\sqrt{t-{\color{cyan}\tau}}}\dd{\color{cyan}\tau}
\end{align*}

Do a change of variables,
\begin{align*}
  z = \sqrt{t-\tau}&\implies z^2 = t - \tau \implies \tau = t-z^2\\
                   &\implies \dd{z} =
                     -\frac{1}{2\sqrt{t-\tau}}\dd{\tau} =
                     -\frac{1}{2z}\dd{\tau}\\
  &\implies \dd{\tau} = -2z\dd{z}\\
                   &\implies v(\tau,x) =
                     v(\tau(t,z),x) = v(t-z^2,x)\\
                   &\implies \pdv{v}{z} = \pdv{v}{\tau}\pdv{\tau}{z}+
                     \pdv{v}{x}\pdv{x}{z} = \pdv{v}{\tau}\cdot (-2z) +
                     \pdv{v}{x}\cdot{0} = -2z\pdv{v}{\tau}\\
                   &\implies \pdv{v}{\tau} = -\frac{1}{2z}\pdv{v}{z}
\end{align*}
Then,
\begin{align*}\frac{1}{\sqrt{\pi}}\int_0^t
  \frac{v_{\tau}({\tau},x)}{\sqrt{t-{\tau}}}\dd{\tau}
 &= \frac{1}{\sqrt{\pi}}\int_{\sqrt{t-0}}^{\sqrt{t-t}}
   \frac{-\frac{1}{2z} v_{z}(t-z^2,x)}{z}(-2z)\dd{z}\\
  &= -\frac{1}{\sqrt{\pi}}\int_{0}^{\sqrt{t}}
    \frac{v_{z}(t-z^2,x)}{z}\dd{z}\\
  &= -\frac{1}{\sqrt{\pi}}\int_{0}^{\sqrt{t}}
    v_{z}(t-z^2,x)\dd{\log (z)}
\end{align*}

\end{document}
%%% Local Variables:
%%% mode: latex
%%% TeX-master: t
%%% End:
