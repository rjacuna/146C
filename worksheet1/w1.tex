\documentclass{article}
\usepackage{fontspec}

% Used to embed Sage code in latex
\usepackage{sagetex}

% Math Environment
\usepackage{euler}        % Euler font
\usepackage{amsmath}      % Math macros
\usepackage{amssymb}      % Math symbols
\usepackage{unicode-math} % Unicode support

% Physics Environment
\usepackage{physics}

\usepackage[makeroom]{cancel} % Used to cancel terms in algebraic equations
\usepackage{ulem} % Different underline environments
\usepackage{polynom} %Polynomial long division

% Typesetting Rules
\setlength\parindent{0em}
\setlength\parskip{0.618em}
\usepackage[a4paper,lmargin=1in,rmargin=1in,tmargin=1in,bmargin=1in]{geometry}
\setmainfont[Mapping=tex-text]{Helvetica Neue LT Std 45 Light}

% Common Macros
\newcommand\N{\mathbb{N}}
\newcommand\Z{\mathbb{Z}}
\newcommand\R{\mathbb{R}}
\newcommand\C{\mathbb{C}}
\newcommand\A{\mathbb{A}}
\def\res{\mathop{\text{Res}}\limits}

% Color
\usepackage[dvipsnames]{xcolor}
\usepackage{pagecolor}
\definecolor{DeepCyan}{HTML}{006969}
\definecolor{DeepRed}{HTML}{690000}
\pagecolor{DeepCyan}
\color{Goldenrod}



\begin{document}

\begin{center}
  146C --- Worksheet 1

  RJ Acuña

  (862079740)
\end{center}\vspace{1.618em}

\subsection*{Problem 1.} Use the mean value theorem to show that there is a unique solution to the initial
value problem \[\begin{cases}x'(t) = 0 \\ x(t_0) = c \end{cases}\]

\uwave{pf. }

Let $[a,b]\subset [R],$ since $x'(t)= 0\implies x(t)$ is
differentiable. So, by MVT $\exists d\in (a,b):$\[x(b)-x(a) =
  x'(d)(b-a) = 0 \implies x(b)= x(a) \implies x(t)\text{ is constant.}\]
\[x(t_0)= c\implies x(t)= c\quad \blacksquare\]

\subsection*{Problem 2.} Suppose $u$ is a function of $(t,x)$ and that
\[\begin{cases}\pdv{u}{t} = 0 \\ u(0,x) = f(x) \end{cases}\]where $f$ is a continuously differentiable function. Solve for $u$. Take $f (x)$ to be equal to $x^2$ and
then once again solve for $u$.

\uwave{slu.}

\[\pdv{u}{t}= 0 \implies u(t,x) = g(x)\]
\[u(0,x) = f(x)\implies u(t,x) = f(x)\]
\[f(x) = x^2 \implies u(t,x) = x^2\quad \lozenge\]

\subsection*{Problem 3.} Suppose that at every point in the $t-x$ plane, the derivative of the function $u$ in
the direction $\vb{v}$ is zero. Take $\vb{v}$ to be equal to $\langle   a,
b \rangle $.

(a) Let $V$ be the constant vector field
\[V(t,x) = \vb{v}\text{ .}\]
Find all curves in the plane that are integral curves of $V$.

\uwave{slu.}
\[\dv{x}{t} = \frac{b}{a}\implies x = \frac{b}{a}t + C \implies ax
  - bt  = C \]
\newpage
(b) Find a system of coordinates $(τ, y)$ so that,
\[\pdv{f}{y} = \vb{v}\cdot \grad{f}\]

\uwave{slu.}

\[\vb{v}\cdot \grad{f} = \langle a,b \rangle\cdot \langle
  \pdv{f}{t},\pdv{f}{x} \rangle =a\pdv{f}{t}+b\pdv{f}{x}\]

\begin{equation} \pdv{f}{t} = {\pdv{f}{\tau}} {\color{green} \pdv{\tau}{t}} +{\pdv{f}{y}} {\color{green}\pdv{y}{t}} =
  {\color{green}\pdv{\tau}{t}}\pdv{f}{\tau}
  +{\color{green}\pdv{y}{t}}\pdv{f}{y}
\end{equation}
\begin{equation}
  \pdv{f}{x} = {\pdv{f}{\tau}}{\color{green}\pdv{\tau}{x}}
  +{\pdv{f}{y}}{\color{green} \pdv{y}{x}} =
  {\color{green}\pdv{\tau}{x}}\pdv{f}{\tau}
  +{\color{green}\pdv{y}{x}}\pdv{f}{y}
  \end{equation}

If ${\color{red} a} = {\color{red}b} = 0$, then the equation imposes a
condition on $f$, that is that $f$ must be a constant in $y$.

If ${\color{red}a} = 0$, then
\[\vb{v}\cdot \grad{f} = b\pdv{f}{x}\]

Let $\tau = -bt$

\[\pdv{f}{t}
  \overset{(1)}{=} {\color{cyan} -b}\pdv{f}{\tau}
  +{\color{green}\pdv{y}{t}}\pdv{f}{y}\quad\text{ and }\quad\pdv{f}{x} \overset{(2)}{=} {\color{red} 0} + {\color{green}
    \pdv{y}{x}}\pdv{f}{y}\]
\[\implies {\color{red} 0}\pdv{f}{t} + {\color{green}b}\pdv{f}{x} ={\color{green}
    b\pdv{y}{x}}\pdv{f}{y} \]
Let $y = {\color{cyan}\frac{1}{b}}x$, then
\[\vb{v}\cdot\grad{f} = {\color{green}
    b\pdv{y}{x}}\pdv{f}{y} ={\color{green}
    b}{\color{cyan}\frac{1}{b}}\pdv{f}{y}= \pdv{f}{y}\]

If ${\color{red}b} = 0$, then
\[\vb{v}\cdot \grad{f} = a\pdv{f}{t}\]
Let $\tau = ax$

\[\pdv{f}{t} \overset{(1)}{=} {\color{red} 0}
  +{\color{green}\pdv{y}{t}}\pdv{f}{y}\quad \text{ and }\quad\pdv{f}{x} \overset{(2)}{=}  {\color{cyan} a}\pdv{f}{\tau} + {\color{green}
    \pdv{y}{x}}\pdv{f}{y}\]
\[\implies {\color{green} a}\pdv{f}{t} + {\color{red} 0}\pdv{f}{x} ={\color{green}
    a\pdv{y}{t}}\pdv{f}{y} \]
Let $y = {\color{cyan}\frac{1}{a}}t$, then
\[\vb{v}\cdot\grad{f} = {\color{green}
    a\pdv{y}{t}}\pdv{f}{y} ={\color{green}
    a}{\color{cyan}\frac{1}{a}}\pdv{f}{y}= \pdv{f}{y}\]

If $a\neq 0$, and $b\neq 0$ Let $\tau = ax - bt,$
\[\pdv{f}{t} \overset{(1)}{=}  {\color{cyan} -b}\pdv{f}{\tau}
  +{\color{green}\pdv{y}{t}}\pdv{f}{y}\quad \text{ and }\quad \pdv{f}{x} \overset{2}{=}  {\color{cyan} a}\pdv{f}{\tau} +{\color{green}
    \pdv{y}{x}}\pdv{f}{y}\]
So,
\[{\color{green}a}\pdv{f}{t} +{\color{green} b}\pdv{f}{x} =
  {\color{green}a}\left(  {\color{cyan} -b}{\color{white}\pdv{f}{\tau}}
    +{\color{green}\pdv{y}{t}}\pdv{f}{y}\right) +{\color{green}
    b}\left(  {\color{cyan} a}{\color{white} \pdv{f}{\tau}}
    +{\color{green}\pdv{y}{x}}\pdv{f}{y}\right) =
  \left({\color{red}a}{\color{green}\pdv{y}{t}}
    +{\color{red}b}{\color{green}\pdv{y}{x}}\right)\pdv{f}{y}\]
We want,\[\left({\color{blue}a}{\color{green}\pdv{y}{t}}
    +{\color{blue}b}{\color{green}\pdv{y}{x}}\right) = 1\]
Let $y = {\color{cyan} \frac{1}{b}}x$, then
\[\vb{v}\cdot \grad{f} = \left({\color{blue}a}{\color{green}\pdv{y}{t}}
    +{\color{blue}b}{\color{green}\pdv{y}{x}}\right) \pdv{f}{y} =
  \pdv{f}{y} \hspace{7em}\blacklozenge\]
\newpage
(c) What curves are the level sets of the $y$ coordinate and how do these curves intersect the
integral curves of $V$?

\uwave{slu.}
That is if $b\neq 0$, the level sets of the $y$ coordinate are $\frac{1}{b}x = D \iff x =
bD$. And they are vertical lines in the $t-x$ plane that intersect $x =
\frac{b}{a}t+C$ at one point. Otherwise, the level sets
are $\frac{1}{a}t = K\iff t = aK$ are  horizontal lines, that
intersect the vertical lies $x = C$ at right angles$\quad\lozenge$

\subsection*{Problem 4.} Suppose that $a$ and $b$ are constants. Solve
the equation
\[au_t+bu_x = 0\text{ with } u(0,x) = f(x)\]

\uwave{slu.}
\[au_t+bu_x = 0\iff \langle a,b \rangle\cdot \grad{u} = 0\]

Then $\grad{u}$ is constant in the direction of $\langle a,b
\rangle$. So, $u(t,x) = g(ax - bt)$.

If $u(0,x) = f(x)$, then $g(ax) =
f(x) \implies u(t,x) = f(\frac{ax-bt}{a})\quad \blacklozenge$

\subsection*{Problem 5.} Suppose that $u$ is a function on $\R$. Solve the equation
\[u' + 3u = 0.\]

\uwave{slu.}
\[u' + 3u = 0\iff u' = -3u\]

The characterizing property of the exponential function is that it is
its own derivative.

Ass. $u = e^{f(t)}$,
\[u = e^{f(t)} \implies u' = e^{f(t)}f'(t) \implies f'(t) = -3
  \implies f(t) = -3t + C\]
\[\implies u = e^{-3t+C} = Ke^{-3t} \implies u' =  -3Ke^{-3t}\]
\[-3Ke^{-3t} + 3Ke^{-3t}= 0 \implies u'+3u =0 \text{ as
    required}\]
So, $u = Ke^{-3t}$ solves the equation$\quad \lozenge$

Additionally, solve the inhomogenous problem
\[u' + 3u = t+1.\]

\uwave{slu.}
\setcounter{equation}{0}

\begin{equation} (fg)' = f'g+fg'\end{equation}

If \begin{equation}g'= hg \iff g'-hg = 0\end{equation}, then
\begin{equation}f'g+fg' = f'g+fhg = g(f'+hf)\end{equation}

The first part hints that (2), has solution\[g(t) = e^{\int_{t_0}^t h(t)
  dt}
\implies g'(t) = e^{\int_{t_0}^t h(t)
  dt}\left( \int_{t_0}^t h(t)
  dt \right)' \overset{FTC 1}{=} h(t)e^{\int_{t_0}^t h(t)
  dt}\]
\[h(t)e^{\int_{t_0}^t h(t)
    dt}-h(t)e^{\int_{t_0}^t h(t)
    dt} = 0\]
\newpage

Therefore, \[g(t)(f'(t)+h(t)f(t)) = e^{\int_{t_0}^t h(t) dt}(f'(t) +h(t)f(t) = \left(
  f(t)e^{\int_{t_0}^t h(t) dt}\right)'\]

Let $h(t) = 3$, then $g(t) = e^{\int_{0}^t 3dt} = e^{3t}$. Let $f(t)=
u(t)$, then
\[  u'(t)+3u(t) = t+1 \iff e^{3t}(u'(t)+3u(t)) = e^{3t}(t+1) \iff
  (u(t)e^{3t})' = e^{3t}(t+1)\]
\[\implies u(t)e^{3t} = \int e^{3t}(t+1) \dd{t} \implies u(t) =
  e^{-3t}\int  e^{3t}(t+1) \dd{t}\]
\begin{align*}
  \implies u(t) &=  e^{-3t}\int  e^{3t}(t+1) \dd{t} & u= t+1 &\hspace{2em}\dd{v} = e^{3t} \dd{t}\\
                &                                   &\dd{u}= \dd{t}&\hspace{2.3em} v = \frac{1}{3}e^{3t}\\
  \implies u(t) &= e^{-3t} \left( \frac{1}{3}e^{3t}(t+1) -\int
                  \frac{1}{3}e^{3t} \dd{t}\right)&&\\
  \implies u(t) &= e^{-3t} \left(  \frac{1}{3}e^{3t}(t+1)
  -\frac{1}{9}e^{3t} + C\right)&&\\
    \implies u(t) &=  \frac{1}{3}(t+1)
                    -\frac{1}{9} + Ce^{-3t}&&\\
  \implies u(t) &=  \frac{1}{3}t+
                  \frac{2}{9} + Ce^{-3t}&&\\
  \implies u'(t) &=  \frac{1}{3}-3Ce^{-3t}&&\\
\end{align*}
Plugging in,
\[u'(t)+3u(t) = \frac{1}{3}-3Ce^{-3t} + t+
  \frac{2}{3} + 3Ce^{-3t} = t+1\]

So, $u(t) = \frac{1}{3}t+\frac{2}{9}+ Ce^{-3t},$ solves the
inhomogeneous problem$\quad \blacklozenge$

\subsection*{Problem 6.} Solve the equation
\[u_t + 2u_x +15u = 10x +5t +5.\]
\uwave{slu.}

Let $\tau = x-2t,$ and $y = \frac{1}{2}x$, then \[u_t+2u_x = u_y\]

Compute $x = 2y \implies \tau = 2y -2t \implies t =
\frac{2y-\tau}{2}$, then

\begin{align*}u_t + 2u_x +15u = 10x +5t +5
  &\implies u_y +15u = 20y
                                             +5\frac{2y-\tau}{2} +5\\
  &\implies u_y +15u = 25y
    -\frac{5}{2}\tau +5\\
  &\implies e^{15y}\left(   u_y +15u\right) = e^{15y}\left(  25y
    -\frac{5}{2}\tau +5\right)\\
  &\implies (u\cdot e^{15y})' = e^{15y}\left(  25y
    -\frac{5}{2}\tau +5\right) dy\\
  &\implies u(\tau, y) = e^{-15y} \int e^{15y}\left(  25y
    -\frac{5}{2}\tau +5\right) dy\\
  &\implies u(\tau, y) =  \frac{25}{15}y-\frac{5}{30}\tau
    +\frac{5}{15}-\frac{25}{15^2} +f(\tau)e^{-15y}\\
  &\implies u(\tau, y) =  \frac{5}{3}y-\frac{1}{6}\tau
    +\frac{2}{9} +f(\tau)e^{-15y}
\end{align*}
Therefore,
\begin{align*}
  \implies u(t, x) &=  \frac{5}{3}\cdot\frac{1}{2} x -\frac{1}{6}\left( x-2t \right)
                        +\frac{2}{9} +f(x-2t)e^{-\frac{15}{2}x}\\
  \implies u(t, x) &=  \frac{4}{6} x +\frac{1}{3}t
                        +\frac{2}{9} +f(x-2t)e^{-\frac{15}{2}x}\\
\end{align*}

Consider $h(t,x) = f(x-2t)e^{-\frac{15}{2}x}$
\begin{align*}
  h_t(t,x) &=-2f'(x-2t)e^{-\frac{15}{2}x} \text{ and } h_x(t,x)
                                          =f'(x-2t)e^{-\frac{15}{2}x}
                                          -\frac{15}{2}f(x-2t)e^{-\frac{15}{2}x}\\
  \implies& -2f'(x-2t)e^{-\frac{15}{2}x} +2\left(f'(x-2t)e^{-\frac{15}{2}x}
                                                      -\frac{15}{2}f(x-2t)e^{-\frac{15}{2}x}
  \right) + 15 f(x-2t)e^{-\frac{15}{2}x} = 0
\end{align*}

So, $h(t,x)$, solves the homogeneous equation $u_t +2u_x + 15u = 0.$

\begin{sagesilent}
  var('t,x')
  u = (4/6)*x+(1/3)*t +2/9
  u_t = u.derivative(t)
  u_x = u.derivative(x)
\end{sagesilent}
Now consider, $p(t,x) = \frac{4}{6} x +\frac{1}{3}t
                        +\frac{2}{9}$, computation shows:
\[p_t +2p_x +15 p = \sage{u_t} +2\left( \sage{u_x} \right)+ 15\left(
    \sage{u}\right) = \sage{u_t +2*u_x +15*u}\]

So, $u(t,x) = \frac{4}{6} x +\frac{1}{3}t
+\frac{2}{9} +f(x-2t)e^{-\frac{15}{2}x}$,
solves the equation$\quad \lozenge$

Compare this with Problem 5: It's the same thing.

What happens if we change the inhomogeneous part
to be different. For example, solve the equation
\[u_t + 2u_x + 15u = 2x + t\]

\uwave{slu.}

$x = 2y \implies 2x = 4y$, and $t = \frac{2y -\tau}{2}$, so $2x+t =5y
-\frac{1}{2}\tau$.

Then $u(t,x)= \frac{5}{15}y -\frac{1}{30}\tau -\frac{5}{15^2} =
\frac{1}{3}y-\frac{1}{30}\tau -\frac{1}{45}= \frac{1}{6}x-\frac{1}{30}(x-2t) -\frac{1}{45}=
\frac{2}{15}x + \frac{1}{15}t -\frac{1}{45}$

\[u_t + 2u_x +15u = \frac{1}{15} + \frac{4}{15} + 2x+t -\frac{1}{3} = 2x+t\]

So, $u(t,x) = \frac{2}{15}x + \frac{1}{15}t -\frac{1}{45} + f(x-2t)e^{-\frac{15}{2}x}$ solves the
equation $\quad \lozenge$

\subsection*{Problem 7.} Suppose that $u$ is a function of the pairs
$(t, x)$.

(a) Find the integral curves for the vector field $V$ given by
\[V(t,x) = \langle 1,2t \rangle.\]
\uwave{slu.}
\[\dv{x}{t} = \frac{2t}{1} = 2t\implies x = t^2 + C\quad \lozenge\]
(b) Find  a solution to the homogeneous problem
\[u_t +2tu_x = 0.\]
\uwave{slu.}
$u(t,x) = f(x-t^2)$ solves the homogeneous problem$\quad \lozenge$

\newpage
(c) Find a solution to the homogeneous problem
\[u_t +2tu_x +3u = 0.\]
\uwave{slu.}
Let $\tau = x - t^2$, then

\begin{align*} \pdv{u}{t} &=  -2t\pdv{u}{\tau} + {\color{green} \pdv{y}{t}}\pdv{u}{y}\\
  \pdv{u}{x} &=\hspace{1.62em}\pdv{u}{\tau}+ {\color{green}\pdv{y}{x}}\pdv{u}{y}
\end{align*}
\begin{align*}
  \implies  -2t\pdv{u}{\tau} + {\color{green} \pdv{y}{t}}\pdv{u}{y} +2t\left(
             \pdv{u}{\tau} + {\color{green} \pdv{y}{x}}\pdv{u}{y} \right) &= \bigg( {\color{green} \pdv{y}{t}}+2t{\color{green} \pdv{y}{x}}\bigg)\pdv{u}{y}
  \end{align*}

We want\[\bigg( {\color{green} \pdv{y}{t}}+2t{\color{green}
    \pdv{y}{x}}\bigg) = 1\]

So $y= t\quad$ works. Then,
\[u_t +2t u_x +3u = 0\implies u_y +3u = 0 \iff e^{3y}(u_y +3u) =
  (ue^{3y})'  = 0\iff u(\tau,y) = e^{-3y}\int \dd{y}  = f(\tau)e^{-3y}\]
\[u(t,x) = f(x-t^2)e^{-3t} \implies u_t(t,x) =
  -2tf'(x-t^2)e^{-3t}-3f(x-t^2)e^{-3t}\text{ and }u_x(t,x) =
  f'(x-t^2)e^{-3t}\]
\[\implies -2tf'(x-t^2)e^{-3t}-3f(x-t^2)e^{-3t} +2t f'(x-t^2)e^{-3t}
  +3f(x-t^2)e^{-3t} = 0\]
Therefore, $u(t,x) = f(x-t^2)e^{-3t}$, solves the homogeneous
part$\quad \lozenge$

(c) Find a solution to the inhomogenous problem.
\[u_t +2tu_x +3u = x-t^2.\]
\uwave{slu.}

$x-t^2 = \tau$, therefore a particular solution $p(\tau,y)$ is given by.
\[p(\tau,y) = e^{-3y}\int e^{3y}\cdot\tau \dd{y}  = \frac{\tau}{3}\]

Then, \[p(t,x) = \frac{x-t^2}{3} = \frac{1}{3}x -\frac{1}{3}t^2\]
\[p_t(t,x) +2tp_x(t,x) +3 p(t,x)= -\frac{2}{3}t +\frac{2}{3}t +
  3\frac{x-t^2}{3} = x-t^2\]
So, \[u(t,x) = \frac{x-t^2}{3} + f(x-t^2)e^{-3t} \]

Solves the inhomogeneous problem$\quad \blacklozenge$
\subsection*{Problem 8.}Can you see a pattern developing in how we solve these kinds of problems. When
will it be straightforward to solve such problems and what kind of difficulties arise in solving
problems of this type? Given an example of a problem that should be solvable, but where it
will by technically difficult if not impossible to get a nice solution
in a closed form.

\uwave{ans:}
Yes I can see the pattern now. An example is $a(t,x)u_t +b(t,x)u_x
+c(t,x)u = \sqrt{1-x^4}$, you need a non elementary integral. Also,
$a,b,$ and $c$ are arbitrary functions of $(t,x)$.


\end{document}
%%% Local Variables:
%%% mode: latex
%%% TeX-master: t
%%% End:
