\documentclass{article}
\usepackage{fontspec}

% Used to embed Sage code in latex
\usepackage{sagetex}

% Math Environment
\usepackage{euler}        % Euler font
\usepackage{amsmath}      % Math macros
\usepackage{amssymb}      % Math symbols
\usepackage{unicode-math} % Unicode support

% Physics Environment
\usepackage{physics}


\usepackage[makeroom]{cancel} % Used to cancel terms in algebraic equations
\usepackage{ulem} % Different underline environments
\usepackage{polynom} %Polynomial long division

% Typesetting Rules
\setlength\parindent{0em}
\setlength\parskip{0.618em}
\usepackage[a4paper,lmargin=1in,rmargin=1in,tmargin=1in,bmargin=1in]{geometry}
\setmainfont[Mapping=tex-text]{Helvetica Neue LT Std 45 Light}

% Common Macros
\newcommand\N{\mathbb{N}}
\newcommand\Z{\mathbb{Z}}
\newcommand\R{\mathbb{R}}
\newcommand\C{\mathbb{C}}
\newcommand\A{\mathbb{A}}
\def\res{\mathop{\text{Res}}\limits}

% Color
\usepackage[dvipsnames]{xcolor}
\usepackage{pagecolor}
\definecolor{DeepCyan}{HTML}{006969}
\definecolor{DeepRed}{HTML}{690000}
\pagecolor{DeepCyan}
\color{Goldenrod}



\begin{document}

\begin{center}
  146C --- 2

  RJ Acuña

  (862079740)
\end{center}\vspace{1.618em}
\subsection*{I.3}
\paragraph{5} Derive the equation of one-dimensional diffusion in a medium that is
moving along the $x$ axis to the right at constant speed $V$.

\uwave{slu.}

The mass function of the dye is given by the integral of the concentration
between the start $a$ up to some $x > a$,

\[M(x,t) = \int_a^x u(y,t) \dd{y} \implies \pdv{M}{t}(x,t) = \int_a^x u_t(y,t) \dd{y}\]

Fick's law says that the mass in this section of pipe cannot change except by flowing in or
out of its ends. So,

\begin{align*}
  &\dv{M}{t} = \text{flow in} -\text{flow out}\\
  &\text{ Since the rod is moving with constant speed } V \text{ to
    the right, we have,}\\
  &\text{flow in} = ku_x(x,t) -Vu(x,t)\text{ and }\text{ flow out }=
    ku_x(a,t) -Vu(a,t)\\
  \implies & \int_a^x u_t(y,t) dy =  ku_x(x,t) -Vu(x,t) -(ku_x(a,t)
             -Vu(a,t))\\
  \implies & \dv{x}\int_a^x u_t(y,t) dy = \dv{x}\left(   ku_x(x,t) -Vu(x,t) -(ku_x(a,t)
             -Vu(a,t))\right)\\
  \overset{\text{FTC 1}}{\implies}
  & u_t(x,t)  = ku_{xx}(x,t) -Vu_{x}(x,t) -(k\cdot0         -V\cdot0)\\
  \implies & u_t = ku_x -Vu_{xx} \quad \blacksquare
\end{align*}
\paragraph{6} Consider heat flow in a long circular cylinder where the temperature
depends
 only on t and on the distance r to the axis of the cylinder. Here
$r =\sqrt{x^2 + y^2}$ is the cylindrical coordinate. From the three-dimensional
heat equation derive the equation $u_t = k(u_{rr} + u_r/r)$.

\uwave{slu. }

The heat equation is 1.3 equation (10)
\[c\rho \pdv{u}{t} = \nabla\cdot \left( \kappa \grad{u} \right)\iff u_t =\frac{\kappa}{c\rho} \laplacian{u} \]

We have a cylindrical coordinate system,
\[\begin{cases}x = \r \cos \theta\\ y = r\sin \theta\\ z =
    z\end{cases} \iff \begin{cases} r = \sqrt{x^2 + y^2}\\ \theta =
    \atan(\frac{y}{x})\\ z = z\end{cases}\]

\[\mathcal{J} =
  \begin{pmatrix} \pdv{x}{r}&\pdv{y}{r}&\pdv{z}{r} \\
    \pdv{x}{\theta}&\pdv{y}{\theta}&\pdv{z}{\theta} \\
  \pdv{x}{z}&\pdv{y}{z}&\pdv{z}{z} \end{pmatrix} =   \begin{pmatrix}
  \cos \theta&\sin \theta & 0 \\
    -r\sin \theta & r\cos \theta & 0 \\
    0&0&1 \end{pmatrix}\]

Note, $|\mathcal{J}| = \begin{vmatrix}
  \cos \theta&\sin \theta \\
    -r\sin \theta & r\cos \theta  \end{vmatrix} = r$, then
  \begin{align*}
    \mathcal{J}^{-1} &= \frac{1}{r} \begin{pmatrix}
      \begin{vmatrix}r\cos \theta
        &0\\0&1\end{vmatrix}&\begin{vmatrix}0 &\sin
        \theta\\1&0\end{vmatrix} & \begin{vmatrix}\sin \theta  &
        0\\r\cos \theta & 0\end{vmatrix}\\
      \begin{vmatrix}0
        &-r\sin\theta\\1&0\end{vmatrix}& \begin{vmatrix} \cos \theta &
        0 \\0 & 1\end{vmatrix}&\begin{vmatrix}0 &\cos\theta\\0&-r\sin \theta\end{vmatrix} \\
      \begin{vmatrix}-r\sin \theta &r\cos
        \theta\\0&0\end{vmatrix}&\begin{vmatrix}\sin \theta &\cos
        \theta \\0&0\end{vmatrix}&\begin{vmatrix}\cos \theta
        &\sin\theta\\-r\sin \theta &r\cos
        \theta\end{vmatrix}\end{pmatrix}\\
 &=\frac{1}{r}\begin{pmatrix}
      r\cos \theta
      & -\sin
      \theta & 0\\
      r\sin \theta &\cos \theta & 0\\
      0&0&r\end{pmatrix} = \begin{pmatrix}
      \cos \theta
      & -\frac{1}{r}\sin
      \theta & 0\\
      \sin \theta & \frac{1}{r}\cos \theta & 0\\
      0&0&1\end{pmatrix}
\end{align*}

Then we can see that, \[\mathcal{J}^{-1} = \begin{pmatrix} \pdv{r}{x}&\pdv{\theta}{x}&\pdv{z}{x} \\
    \pdv{r}{y}&\pdv{\theta}{y}&\pdv{z}{y} \\
  \pdv{r}{z}&\pdv{\theta}{z}&\pdv{z}{z} \end{pmatrix}\]

And from the chain rule we can conclude,
\[\pdv{}{z} = 0\pdv{}{r}+0\pdv{}{\theta}+1\pdv{}{z} = \pdv{}{z} \text{
    lol! }\]

Now from the calculations in page 157, we can conclude after squaring
the preceding operator that,
\[\laplacian = \pdv[2]{}{r} +\frac{1}{r}\pdv{}{r}
  +\frac{1}{r^2}\pdv[2]{}{\theta} +\pdv[2]{}{z}\]

Since $u$ depends only on $t$ we have,

\[\pdv[2]{u}{\theta} = 0 \text{ and } \pdv[2]{u}[z]= 0\implies
  \laplacian{u} = \pdv[2]{u}{r} +\frac{1}{r}\pdv{u}{r}\]

Let $k= \frac{\kappa}{c\rho}$ and the result follows,
\[u_t = k(u_{rr} + u_r/r)\quad \blacksquare\]
\paragraph{7} Solve Exercise 6 in a ball except
 that the temperature depends
 only on the spherical coordinate $\sqrt{x^2 + y^2 + z^2}$ . Derive the equation
 $u t = k(u_{rr} + 2u_r /r )$.

 \uwave{slu.}

 Chapter 6.1 formula (6) gives us $\laplacian$ in spherical
 coordinates,
 \[\laplacian u = u_{rr} +\frac{2}{r}u_r +\frac{1}{r^2}[u_{\theta\theta}
   +(\cot{\theta}) u_\theta +\frac{1}{\sin^2{\theta}} u_{\phi\phi}]\]

 Since $u$ only depends on $t$ and $r$ we have,
 \[u_\theta = 0 \text{ and }u_\phi = 0 \implies \laplacian u = u_{rr}
   +\frac{2}{r}u_r\]

 Let $k= \frac{\kappa}{c\rho}$ and the result follows,
\[u_t = k(u_{rr} + 2u_r/r)\quad \blacksquare\]
\paragraph{10} If $\bold{f(x)}$ is continuous and $|\vb{f(x)}|\leq
\frac{1}{|\bold{x}|^3+1}$ for all $\bold{x}$ show that
\[\iiint_{\text{all space}} \nabla\cdot \bold{f} \dd{s} = 0\]

(\textit{Hint:} Take $D$ to be a large ball, apply the divergence
theorem, and let its radius tend to infinity)

\uwave{slu.}
Let $B_R = \{\vb{x}| |x| < R\}$ and bdy $B_R = S_R =\{\vb{x}\in \R^3
|\quad |x|=R\} $
\begin{align*}
  \bigg|\iiint_{B_R} \nabla\cdot \bold{f} \dd{s}\bigg|
  &\overset{\text{div. thm.}}{=}
    \bigg|\iint_{S_R} \bold{f}
    \cdot \vb{n} \dd{s}
    \bigg| \\
  &\leq \iint_{S_R} |\bold{f}
    \cdot \vb{n}|
  ds \leq \iint_{S_R} |\bold{f}|
    |\vb{n}| \dd{s} \leq \iint_{S_R} |\bold{f}|
    1 \dd{s}\\
  &\leq \iint_{S_R} |\bold{f}| \dd{s} \leq \iint_{S_R} \bigg|\frac{1}{|\bold{x}|^3+1}\bigg| \dd{s}
    = \iint_{S_R} \frac{1}{R^3+1} \dd{s}\\
  &= \frac{1}{R^3+1}\iint_{S_R} \dd{s} =\frac{4\pi R^2}{ R^3 +1 }\\
\end{align*}
\[\iiint_{\text{all space}} \nabla\cdot \bold{f} \dd{s} =
  \lim_{R\rightarrow \infty} \frac{4\pi R^2}{ R^3 +1 } =
  \lim_{R\rightarrow \infty} \frac{4\pi}{ R +\frac{1}{R^2}} = 0\quad \blacksquare\]
\paragraph{11} If curl $\bold{v} = 0$ in all of three-dimensional space, show that there exists a
scalar function$ \phi(x, y, z)$ such that $v =$ grad $\phi$.

\uwave{slu. }
Suppose curl $\vb{v} = \vb{0}$,

\begin{align*}
  \curl{\vb{v}}
  &= \begin{vmatrix} \vb{i} &\vb{j} &\vb{k}\\
    \pdv{}{x} &\pdv{}{y} &\pdv{}{z}\\
    v_1 &v_2 &v_3\end{vmatrix}\\
  &= \left(\pdv{v_3}{y} -\pdv{v_2}{z}, -\left( \pdv{v_3}{x}-
    \pdv{v_1}{z} \right), \pdv{v_2}{x} -\pdv{v_1}{y} \right)\\
  &= \left(\pdv{v_3}{y} -\pdv{v_2}{z},
    \pdv{v_1}{z} -\pdv{v_3}{x}, \pdv{v_2}{x} -\pdv{v_1}{y} \right)=
    \vb{0}\\
 \implies& \begin{cases} \pdv{v_3}{y} -\pdv{v_2}{z} = 0\\ \pdv{v_1}{z}
   -\pdv{v_3}{x} = 0 \\ \pdv{v_2}{x} -\pdv{v_1}{y} = 0\end{cases}
  \implies \begin{cases}\color{green} \pdv{v_3}{y} = \pdv{v_2}{z} \\ \color{White}\pdv{v_1}{z}
   = \pdv{v_3}{x} \\\color{Orange} \pdv{v_2}{x} = \pdv{v_1}{y} \end{cases}
\end{align*}

Let $\phi(x,y,z) =\int_0^z v_3(x,y,z) \dd{z}$
\begin{align*}
  \pdv{\phi}{x}  &= \int_0^z \color{White}
                   \pdv{v_3}{x}\color{Goldenrod}\dd{z}
                   = \int_0^z \color{White}
                   \pdv{v_1}{z}\color{Goldenrod}\dd{z} = v_1\\
  \pdv{\phi}{y}  &= \int_0^z \color{green}
                   \pdv{v_3}{y}\color{Goldenrod}\dd{z}
                   = \int_0^z \color{green}
                   \pdv{v_2}{z}\color{Goldenrod}\dd{z} = v_2\\
  \pdv{\phi}{z}  &= \int_0^z \color{cyan}
                   \pdv{v_3}{z}\color{Goldenrod}\dd{z}
                   = v_3\\
  \implies& \vb{v} = \grad{\phi} \quad \blacksquare
\end{align*}
\subsection*{I.4} (Hint: Use equations (4) and (5) on Page 23. For (a), note that the curl of a gradient is
zero.)
\paragraph{1}By trial and error, find a solution of the diffusion equation $u_t = u_{x x}$ with
the initial condition $u(x, 0) = x^2$.

\uwave{slu.}

Assume $u(x,t) = f(x) + g(t)$,

\[u(x,0) = f(x) +g(0) = x^2 \implies f(x) = x^2\text{ and
  }g(0) = 0\]
\[u_t = f_t+g_t = f_{xx} +g_{xx} = u_{xx}\]
\[ \implies 0 + g_t = 2  +0  \implies \dv{g}{t} = 2 \implies g(t) = 2t + C \]
\[g(0)=0 \implies C = 0\]
So, \[u(x,t) = x^2 +2t\quad \blacklozenge\]
\paragraph{7} In linearized gas dynamics (sound), verify the following.

(a) If curl $\bold{v} = 0$ at $t = 0$, then curl $\bold{v} = 0$ at all later times.

(b) Each component of $\vb{v}$ and $ρ$ satifies the wave equation.

\uwave{slu.}

Let $\vb{x}\in \R^3$ we know $\curl{\vb{v(0,\vb{x})}} = 0$, we want to
show $\forall t> 0,\, \curl{\vb{v(t,\vb{x})}} = 0$. That is that the
curl doesn't change. So it's enough to show,
\[\dv{t}\curl{\vb{v}}=0\]

Now sound satisfies the following system of equations in page 23,
\[\begin{cases}\color{White} \pdv{\vb{v}}{t} + \frac{c_0^2}{\rho_0}\grad{\rho} = 0\\
    \pdv{\rho}{t} +\rho_0\nabla\cdot \vb{v} = 0\end{cases}\]

Since,
\begin{align*}
  \dv{t}\curl{\vb{v}}
  &=\curl{\color{White}\dv{\vb{v}}{t}}\\
  &=
    \curl{\color{green}-\color{White}\frac{c_0^2}{\rho_0}\grad{\rho}}\\
  &= -\frac{c_0^2}{\rho_0}{\color{SkyBlue}\curl{\grad{\rho}}}\\
  &= -\frac{c_0^2}{\rho_0}{\color{SkyBlue}0} = 0\\
  \implies &\curl{\vb{v}} \equiv C\in \R\\
 \curl{\vb{v}(0,\vb{x})} = 0\implies & C = 0
\end{align*}

So curl $\vb{v}$ is $0$ for all $t>0.$ That completes part (a).
\newpage
For part (b), differentiating with respect to $t$ the system of
eqns. in page 23, should yield the result:

\begin{align*}
  \begin{cases}\pdv{\vb{v}}{t} + {\color{cyan}\frac{c_0^2}{\rho_0}\grad{\rho}} = 0\\
  \pdv{\rho}{t} +{\color{white}\rho_0\nabla\cdot \vb{v}} = 0\end{cases}
  &\implies \begin{cases}\pdv[2]{\vb{v}}{t} + \pdv{}{t}\frac{c_0^2}{\rho_0}\grad{\rho} = 0\\
    \pdv[2]{\rho}{t} +\pdv{t}\rho_0\nabla\cdot \vb{v} = 0\end{cases}\\
  &\implies \begin{cases}\pdv[2]{\vb{v}}{t} + \frac{c_0^2}{\rho_0}\grad{\pdv{\rho}{t}} = 0\\
    \pdv[2]{\rho}{t} +\rho_0\nabla\cdot \pdv{\vb{v}}{t} = 0\end{cases}\\
&\implies \begin{cases}\pdv[2]{\vb{v}}{t} +
  \frac{c_0^2}{\rho_0}\grad{\left(  {\color{green} - \color{white} \rho_0\nabla\cdot \vb{v}}\right)} = 0\\
    \pdv[2]{\rho}{t} +\rho_0\nabla\cdot
    {\color{green}-\color{cyan}\frac{c_0^2}{\rho_0}\grad{\rho}} =
    0\end{cases}\\
  &\implies \begin{cases}\pdv[2]{\vb{v}}{t}  \color{green}
    -\color{Goldenrod} \frac{c_0^2\cancel{\rho_0}}{\cancel{\rho_0}} {\color{white}\nabla\cdot\color{Goldenrod}\grad{\color{white} \vb{v}}} = 0\\
    \pdv[2]{\rho}{t} \color{green}
    -\color{Goldenrod} \frac{\cancel{\rho_0}c_0^2}{\cancel{\rho_0}}
    {\color{cyan}\color{Goldenrod}\nabla \cdot\color{cyan} \grad{\rho}} =
    0\end{cases}\\
  &\implies \begin{cases}\pdv[2]{\vb{v}}{t} = c_0^2\laplacian{\vb{v}}\\
    \pdv[2]{\rho}{t} =c_0^2 \laplacian{\rho}\end{cases}\\
\end{align*}
\subsection*{2.1}
For 3: Assume that the string has an initial square displacement and that waves propagating
in opposite directions are produced. Do not assume an initial velocity as you might considering
it is a hammer strike.
I’d like you to try problems 5 and 6 from this section, but won’t require them for homework.


\paragraph{1} Solve $u_{tt} = c^2 u_{ x x}$, $u(x, 0) = e^x$ , $u_t
(x, 0) = \sin x$.

\uwave{slu. }
\begin{align*}
  u_{tt} = c^2 u_{ x x}
  &\implies u(x,t) = \frac{1}{2}\left(e^{x+ct}+e^{x-ct}\right) +\frac{1}{2c}
                          \int_{x-ct}^{x+ct}\sin x dx\\
  &\implies u(x,t) = \frac{1}{2}\left(e^{x+ct}+e^{x-ct}\right)+
    \frac{1}{2c}\left(  \cos(x+ct)-\cos(x-ct)\right)\\
  &\implies u(x,t) = e^x\left(\frac{e^{ct}+e^{-ct}}{2}\right) +
    \frac{1}{2c}\left(\cos(x+ct)-\cos(x-ct)\right)\\
  &\implies u(x,t) = e^x\cosh ct + \frac{-2}{2c}\sin\left(
    \frac{x+ct+x-ct}{2} \right)\sin\left( \frac{x+ct-(x-ct)}{2}
    \right)\\
    &\implies u(x,t) = e^x\cosh ct -\frac{1}{c}\sin x \sin ct \quad \blacklozenge
\end{align*}

\paragraph{2} Solve $u_{tt} = c^2 u_{x x} , u(x, 0) = \log(1 + x^2),
u_t (x, 0) = 4 + x$.

\uwave{slu.}

\begin{align*}
  u_{tt} = c^2 u_{ x x}
  &\implies u(x,t) = \frac{1}{2}\left( \log (1+(x+ct)^2)
    +\log(1+(x-ct)^2)\right)
    +\frac{1}{2c}
      \int_{x-ct}^{x+ct} 4+x \dd{x}\\
       &\implies u(x,t) = \frac{1}{2}\log ((1+ (x+ct)^2)(1 +(x-ct)^2))
       +4t+xt\\
    &\implies u(x,t) = \log (\sqrt{x^4+2(1-c^2t^2)x^2+(1 + c^2t^2)^2})     +4t+xt \quad \blacklozenge
\end{align*}

\newpage
\paragraph{3} The midpoint of a piano string of tension $T$, density $ρ$, and length $l$ is hit
by a hammer whose head diameter is $2a$. A flea is sitting at a distance
$l/4$ from one end. (Assume that $a < l/4$; otherwise, poor flea!) How long
does it take for the disturbance to reach the flea?

\uwave{slu. }

The hammer hits the midpoint of  the string at $l/2$, since the hammer
head is $2a$, at $t = 0$, the hammer extends from $l/2 -a$ to $l/2
+a$. This generates two waves going at wave speed $c = \sqrt{T/\rho}$.

The wave speed is constant thus $\varDelta x = \sqrt{T/\rho}t$.

The distance between the right side of the string that hits the hammer
to the point $3l/4$ where the flea is sitting is,
\[3l/4 - (l/2 +a) = l/4 - a = \varDelta x\]

\[\implies l/4 -a = \sqrt{T/\rho}t \implies t =
  \sqrt{\rho/T}(l/4-a)\quad \lozenge\]

\paragraph{9} Solve $u_{x x} − 3u_{xt} − 4u_{tt} = 0, u(x, 0) = x^2 ,
u_t (x, 0) = e^x$ . (\textit{Hint:} Factor the operator as we did for
the wave equation.)

\uwave{slu.}

\[u_{x x} − 3u_{xt} − 4u_{tt} = 0 \iff \left( \pdv[2]{x}
    -3\pdv{}{x}{t} -4\pdv[2]{t} \right)u = 0
  \iff \left( \pdv{x}-4\pdv{t} \right)\left( \pdv{x}+\pdv{t} \right) u
  = 0\]

Let $v = \left( \pdv{x} + \pdv{t} \right) u$,

\[\left( \pdv{x}-4\pdv{t}\right)v = 0 \implies v_x -4v_t = 0\implies -v_x +4v_t = 0 \implies v(x,t) = f(4x+t)\]
\[v = \left( \pdv{x} + \pdv{t} \right) u \implies u_x+u_t = f(4x+t)\]
\[u_x(x,0) = 2x\text{ and }u_t(x,0) = e^x \implies 2x+e^x = f(4x)\]

Let $g(s) = \frac{1}{4} s$ and $h(s) = 2s + e^s$

\[f= h\circ g \implies f(4x+t) = h(x+\frac{1}{4}t) = 2x+\frac{1}{2}t
  +e^{x+\frac{1}{4}t} \implies f(4x+0) = 2x+0+e^x+0\]

So, $f$ works, and we have,
\begin{equation}
  u_x+u_t = 2x+\frac{1}{2}t
  +e^{x+\frac{1}{4}t}
\end{equation}

Now the left hand side factors as,\[u_x+u_t = \langle 1,1
  \rangle\grad{u}\]

This suggests the following change of variables, $y= x+t$ and $z =
x-t$ which gives,
\[u_x = u_y+u_z\text{ and }u_t =u_y-u_z\]
\[x = \frac{y+z}{2}\text{ and } t = \frac{y-z}{2}\]
Plugging into (1)
\[u_y+u_z+u_y-u_z= 2u_y = 2\frac{y+z}{2} +\frac{1}{2}\frac{y-z}{2}
  +e^{\frac{y+z}{2}+\frac{1}{4}\frac{y-z}{2}}\]
\[\implies u_y = \frac{y+z}{2} +\frac{1}{4}\frac{y-z}{2}
  +\frac{1}{2}e^{\frac{y+z}{2}+\frac{1}{4}\frac{y-z}{2}}\]
\[\implies u_y = \frac{5}{8}y+\frac{3}{8}z
  +\frac{1}{2}e^{\frac{5}{8}y+\frac{3}{8}z}\]
\[\implies u_y = \frac{5}{8}y+\frac{3}{8}z
  +\frac{1}{2}e^{\frac{3}{8}z}e^{\frac{5}{8}y}\]

Regarding $z$ as fixed we can integrate with respect to $y$,

\begin{align*}
  u(y,z) &= \int \frac{5}{8}y+\frac{3}{8}z
           +\frac{1}{2}e^{\frac{3}{8}z}e^{\frac{5}{8}y}\dd{y}\\
  &= \frac{5}{16}y^2+\frac{3}{8}yz
    +\frac{8}{10}e^{\frac{3}{8}z}e^{\frac{5}{8}y} + k(z)\\
\end{align*}
\begin{equation}  u(y,z)=  \frac{5}{16}y^2+\frac{3}{8}yz
  +\frac{4}{5}e^{\frac{3}{8}z}e^{\frac{5}{8}y} + k(z)
\end{equation}

Then plugging $x$ and $t$ back into (2)
\[u(x,t) = \frac{5}{16}(x+t)^2+\frac{3}{8}(x+t)(x-t)
           +\frac{4}{5}e^{\frac{3}{8}(x-t)}e^{\frac{5}{8}(x+t)} +
           k(x-t)\]

Now simplifying gives
\begin{align*}
  u(x,t) &= \frac{5}{16}(x+t)^2+\frac{3}{8}(x+t)(x-t)
           +\frac{4}{5}e^{\frac{3}{8}(x-t)}e^{\frac{5}{8}(x+t)} +
           k(x-t)\\
  &= \frac{5}{16}(x^2+2xt+t^2)+\frac{3}{8}(x^2-t^2)
           +\frac{4}{5}e^{\frac{3}{8}(x-t)+\frac{5}{8}(x+t)} +
    k(x-t)\\
  &= \frac{11}{16}x^2+\frac{5}{8}xt-\frac{1}{16}t^2
           +\frac{4}{5}e^{x +\frac{1}{4}t} +
           k(x-t)\\
\end{align*}
\begin{equation}
  u(x,t) = \frac{11}{16}x^2+\frac{5}{8}xt-\frac{1}{16}t^2
           +\frac{4}{5}e^{x +\frac{1}{4}t} +
           k(x-t)
  \end{equation}
Now we need $k$ such that $u(x,0)= x^2$, so we plug $0$ into (3)

\begin{align*}
  u(x,0) &= \frac{11}{16}x^2+\frac{5}{8}x0-\frac{1}{16}0^2
           +\frac{4}{5}e^{x +\frac{1}{4}0} +
           k(x-0)\\
  &= \frac{11}{16}x^2
           +\frac{4}{5}e^{x} +
    k(x) = x^2\\
  \implies& k(x) = x^2 - \frac{11}{16}x^2
            -\frac{4}{5}e^{x} \\
  \implies& k(x) = \frac{5}{16}x^2 -\frac{4}{5}e^{x} \\
  \implies& k(x-t) = \frac{5}{16}(x-t)^2 -\frac{4}{5}e^{x-t} \\
\end{align*}

Now we can plug $k(x-t)$ back into (3)

\begin{align*}
  u(x,t) &= \frac{11}{16}x^2+\frac{5}{8}xt-\frac{1}{16}t^2
           +\frac{4}{5}e^{x +\frac{1}{4}t} +
           \frac{5}{16}(x-t)^2 -\frac{4}{5}e^{x-t}\\
  &= \frac{11}{16}x^2+\frac{5}{8}xt-\frac{1}{16}t^2
           +\frac{4}{5}e^{x +\frac{1}{4}t} +
    \frac{5}{16}(x^2-2xt+t^2) -\frac{4}{5}e^{x-t}\\
    &= \left( \frac{11}{16}+\frac{5}{16}\right)x^2+\frac{5}{8}xt-\frac{5}{8}xt+\left( \frac{5}{16}- \frac{1}{16}\right)t^2
           +\frac{4}{5}e^{x +\frac{1}{4}t}
      -\frac{4}{5}e^{x-t}\\
  &= x^2+\frac{1}{4}t^2 +\frac{4}{5}e^{x +\frac{1}{4}t}
    -\frac{4}{5}e^{x-t}\\
\end{align*}
Thus,
\begin{equation}
  u(x,t) = x^2+\frac{1}{4}t^2 +\frac{4}{5}e^{x-t}\left(  e^{\frac{5}{4}t}-1\right)
\end{equation}
\begin{sagesilent}
  var('x,t')
  u = x^2 +(1/4)* t^2 +(4/5) *(e^(5*t/4)-1) * e^(x-t)
  u_xx = u.derivative(x,2)
  u_xt = u.derivative(x).derivative(t)
  u_tt = u.derivative(t,2)
\end{sagesilent}

We get $u(x,0)= x^2$ and $u_t(x,0)= e^x$, and,

\[u_{xx} =\sage{u_xx}\text{ and }u_{xt} = \sage{u_xt}\text{ and
  }u_{tt} = \sage{u_tt} \]
\[u_{xx} -3u_{xt}-4u_{tt} = \sage{u_xx-3*u_xt -4*u_tt}\]



% So the change of variables didn't work...

% Let's try it the other way around.
% \[\left( \pdv{x}-4\pdv{t} \right)\left( \pdv{x}+\pdv{t} \right) u
%   = 0 \implies \left( \pdv{x}+\pdv{t} \right)\left( \pdv{x}-4\pdv{t} \right) u
%   = 0\]

% Let $v = u_x-4u_t$, then

% \[v_x+v_t = 0 \implies v(x,t) = f(x-t) \implies 2x+e^x=f(x)\implies
%   f(x-t) = 2(x-t)+e^{x-t}\]
% \[\implies u_x-4u_t = 2(x-t)+e^{x-t}\]

% Let $y = x-4t$ and $z = -4x -t$ then,
% \[x = \frac{y-4z}{17}\text{ and } t = -\frac{4y+z}{17}\implies x-t =
%   \frac{-3y-5z}{17} = -\frac{3y+5z}{17}\]
% \[u_x = u_y-4u_z\text{ and }u_t = -4u_y-u_z\implies u_y-4u_z
%   -4(-4u_y-u_z) = 17u_y = 2\left(-\frac{3y+5z}{17}\right)+
%   e^{-\frac{3y+5z}{17}}\]
% Then,
% \[u_y = \frac{2}{17}\left(-\frac{3y+5z}{17}\right)+
%   \frac{1}{17} e^{-\frac{3y+5z}{17}}\]
% \begin{sagesilent}
%   var('y,z')
%   u = -1*(2/17)*(3*y+5*z)/17 + (1/17)*exp(-1*(1/17)*(3*y+5*z))
%   iu = u.integral(y)
% \end{sagesilent}

% We can regard $z$ as fixed and integrate,

% \[u(y,z) = \sage{iu} + k(z)\]

% Now substitute $x$ and $t$ back in,
% \[u(x,t) = \sage{iu.subs(y= x-4*t).subs(z = -4*x-1*t) } + k(-4x-t)\]
% \[\implies u(x,t) = \sage{iu.subs(y= x-4*t).subs(z =
%     -4*x-1*t).full_simplify() } + k(-4x-t)\]
% It has to satisfy $u(x,0)= x^2$, solution
% \[\implies u(x,t) = \sage{iu.subs(y= x-4*t).subs(z =
%     -4*x-1*t).full_simplify().subs(t=0) } + k(-4x) = x^2\]
% \[\implies k(-4x) = \sage{(x^2 - iu.subs(y= x-4*t).subs(z =
%     -4*x-1*t).full_simplify().subs(t=0)).full_simplify() }\]
% \begin{sagesilent}
%   var('s')
%   m = (252/289)*s^2 +(1/3)*exp(s)
%   l = -1*(1/4)*s
%   k = m.subs(s = l.subs(s=-4*x-t))
%   fu = iu.subs(y= x-4*t).subs(z = -4*x-1*t).full_simplify() +k
% \end{sagesilent}
% Let $l(s) = -\frac{1}{4}s$ and $m(s) = \sage{m}$,
% \[k(-4s) = (m\circ l)(-4s) = \sage{m.subs(s = l.subs(s=-4*s))} \]
% \[\implies k(-4x-t) = (m\circ l)(-4s) = \sage{m.subs(s =
%     l.subs(s=-4*x-t))} \]

% \[\implies  u(x,t) = \sage{fu}\]
% \[\implies  u(x,t) = \sage{fu.full_simplify()}\]
% \begin{sagesilent}
%   u_t = fu.derivative(t)
%   u_xx = fu.derivative(x,2)
%   u_xt = fu.derivative(x).derivative(t)
%   u_tt = fu.derivative(t,2)
% \end{sagesilent}
% Again we have $u(x,0)= \sage{fu.subs(t=0)}$ and $u_t(x,0) = \sage{u_t.subs(t=0)}$
% \[u_{xx} -3u_{xt}-4u_{tt} = \sage{u_xx-3*u_xt -4*u_tt}\]

% I must have made a mistake, now let's try adjusting such  that
% $u_t(x,0)= e^x$
% \[u(x,t) = x^2 -\frac{1}{4}t^2 -\frac{4}{3}\left( e^{\frac{1}{4}t}-e^t
%   \right)e^x\]
% \begin{sagesilent}
%   ffu = x^2 -1*(1/4)*t^2 -1*(4/3)*(exp((1/4)*t)-exp(t))*exp(x)
%   u_t = ffu.derivative(t)
%   u_xx = ffu.derivative(x,2)
%   u_xt = ffu.derivative(x).derivative(t)
%   u_tt = ffu.derivative(t,2)
% \end{sagesilent}

% Now we have $u(x,0)= \sage{fu.subs(t=0)}$ and $u_t(x,0) = \sage{u_t.subs(t=0)}$
% \[u_{xx} -3u_{xt}-4u_{tt} = \sage{(u_xx-3*u_xt -4*u_tt).full_simplify()}\]

% Which is arguably worse...


% I guess at this point we can guess that solutions will be of the form,
% \[u(x,t) = x^2 +f(t) +\frac{e^x}{C}(e^{Ct}-g(t)): f(0) = 0 \text{ and
%   } g(0) = 1\]

% \[u_t(x,t) = f'(t) +e^x(e^{Ct} -g'(t)): f'(0) = 0 \text{
%     and }g'(0) = 0\]

% \[u_{xx} = 2 +\frac{e^x}{C}(e^{Ct}-g(t))\text{ and }u_{tx}= u_{xt} =
%  e^x(e^{Ct} -g'(t)) \text{ and }u_{tt}  = f''(t) +Ce^x(e^{Ct}
%  -g''(t))\]
% \[\implies 2 +\frac{e^xe^{Ct}}{C}-\frac{e^xg(t)}{C} -3e^x(e^{Ct} -g'(t))-4(f''(t) +Ce^x(e^{Ct}
%   -g''(t))) = 0 \]
% \[\implies 2 +\frac{e^xe^{Ct}}{C}-\frac{e^xg(t)}{C} -3e^xe^{Ct} +3e^xg'(t)-4f''(t) -4Ce^xe^{Ct}
%   +4Cg''(t) = 0 \]
% \[\implies 2 +3e^xg'(t)-4f''(t)
%   +4Cg''(t) -\frac{e^xg(t)}{C}
%   -3e^xe^{Ct}+\frac{e^xe^{Ct}}{C}-4Ce^xe^{Ct} = 0 \]
% \[\implies 2 -4f''(t) +4Cg''(t) +3e^xg'(t)
%   -\frac{e^xg(t)}{C} +\left(\frac{1}{C}-3-4C\right)e^xe^{Ct} = 0 \]

% Since $e^{x}e^{Ct}\neq 0$, then
% \[\frac{1}{C}-3-4C = 0\implies 4C^2 +3C -1 = 0 \implies C =
%   \frac{-3\pm\sqrt{9-4(4)(-1)}}{2} \implies C = 1 \text{ or } C = -4\]

% Nope... This is definitely not working...

% Slader said that

% \[u(x,t)= f(4x+t)\text{ or } u(x,t) = g(x-t)\]

% And since the equation is linear you can take $u$ to be the sum of
% both solutions

% \[u(x,t) = f(4x+t)+g(x-t)\]

% They computed
% \[u(x,t) = x^2+\frac{1}{4}t^2 +\frac{4}{5}e^x\left( e^{\frac{3}{4}t} -1\right)\]
% Chegg to the rescue...

% \[u_{x x} − 3u_{xt} − 4u_{tt} = 0 \iff \left( \pdv[2]{x}
%     -3\pdv{}{x}{t} -4\pdv[2]{t} \right)u = 0
%   \iff \left( \pdv{x}-4\pdv{t} \right)\left( \pdv{x}+\pdv{t} \right) u
%   = 0\]

% The general form of the that equation is the following, and it is
% solvable  if $ad-bc\neq 0$
% \[\begin{cases}\left( a\pdv{x}+b\pdv{t} \right)\left( c\pdv{x}+d\pdv{t} \right) u
%   = 0 \\ u(x,0) =\phi(x) \text{ and  }u_t(x,0) = \psi(x)\end{cases}\]
% And the solution is given by\[ u(x,t) = \frac{bd}{ad-bc}\left(
%     \int_0^{x-\frac{c}{d}t} \psi(y) \dd{y} -\int_0^{x-\frac{a}{b}t}
%     \psi(y) \dd{y} \right) +
%   \frac{ad}{ad-bc}\phi(x-\frac{c}{d}t)-\frac{bc}{ad-bc}\phi(x-\frac{a}{b}t)\]

% Substituting $\psi(x) = e^x$ and $\phi(x) = x^2$ we get,

% \begin{sagesilent}
%   from sage.symbolic.integration.integral import definite_integral
%   var('a,b,c,d,x,t,y')
%   u = ((b*d*(definite_integral(exp(y),y,0, x-(c/d)*t)-definite_integral(exp(y),y,0,x-(a/b)*t)) +a*d*(x-(c/d)*t)^2-b*c(x-(a/b)*t)^2)/(a*d-b*c))
% \end{sagesilent}
% \[u(x,t) = \sage{u}\]
% Substituting $a =1, b=-4,c=1,d=1$ we get,
% \begin{sagesilent}
%   u = u.subs(a=1,b=-4,c=1,d=1)
%   u_xx = u.derivative(x,2)
%   u_xt = u.derivative(x).derivative(t)
%   u_tt = u.derivative(t,2)
% \end{sagesilent}
% \[u(x,t) = \sage{u}\]
% \[u_{xx} =\sage{u_xx}\]\[\text{ and }u_{xt} = \sage{u_xt}\]\[\text{ and
%   }u_{tt} = \sage{u_tt} \]
% \[u_{xx} -3u_{xt}-4u_{tt} = \sage{u_xx-3*u_xt -4*u_tt}\]

% Furthermore, $u(x,0)= \sage{u.subs(t=0)}$ and $u_t(x,0)= \sage{u.derivative(t).subs(t=0)}$.

% I want to simplify a little though,
% \[u(x,t) = \frac{1}{20}(t^2 + 8xt + 16x^2) +\frac{1}{5}(t^2-2xt+x^2) +\frac{4}{5}e^{x-t}\left( e^{\frac{5}{4}t} -1\right)\]

% \[u(x,t) = \frac{1}{4}t^2 +x^2 +\frac{4}{5}e^{x-t}\left( e^{\frac{5}{4}t} -1\right)\]

\end{document}
%%% Local Variables:
%%% mode: latex
%%% TeX-master: t
%%% End:
